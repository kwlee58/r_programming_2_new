% Options for packages loaded elsewhere
\PassOptionsToPackage{unicode}{hyperref}
\PassOptionsToPackage{hyphens}{url}
\documentclass[
]{article}
\usepackage{xcolor}
\usepackage[margin=1in]{geometry}
\usepackage{amsmath,amssymb}
\setcounter{secnumdepth}{-\maxdimen} % remove section numbering
\usepackage{iftex}
\ifPDFTeX
  \usepackage[T1]{fontenc}
  \usepackage[utf8]{inputenc}
  \usepackage{textcomp} % provide euro and other symbols
\else % if luatex or xetex
  \usepackage{unicode-math} % this also loads fontspec
  \defaultfontfeatures{Scale=MatchLowercase}
  \defaultfontfeatures[\rmfamily]{Ligatures=TeX,Scale=1}
\fi
\usepackage{lmodern}
\ifPDFTeX\else
  % xetex/luatex font selection
\fi
% Use upquote if available, for straight quotes in verbatim environments
\IfFileExists{upquote.sty}{\usepackage{upquote}}{}
\IfFileExists{microtype.sty}{% use microtype if available
  \usepackage[]{microtype}
  \UseMicrotypeSet[protrusion]{basicmath} % disable protrusion for tt fonts
}{}
\makeatletter
\@ifundefined{KOMAClassName}{% if non-KOMA class
  \IfFileExists{parskip.sty}{%
    \usepackage{parskip}
  }{% else
    \setlength{\parindent}{0pt}
    \setlength{\parskip}{6pt plus 2pt minus 1pt}}
}{% if KOMA class
  \KOMAoptions{parskip=half}}
\makeatother
\usepackage{color}
\usepackage{fancyvrb}
\newcommand{\VerbBar}{|}
\newcommand{\VERB}{\Verb[commandchars=\\\{\}]}
\DefineVerbatimEnvironment{Highlighting}{Verbatim}{commandchars=\\\{\}}
% Add ',fontsize=\small' for more characters per line
\usepackage{framed}
\definecolor{shadecolor}{RGB}{248,248,248}
\newenvironment{Shaded}{\begin{snugshade}}{\end{snugshade}}
\newcommand{\AlertTok}[1]{\textcolor[rgb]{0.94,0.16,0.16}{#1}}
\newcommand{\AnnotationTok}[1]{\textcolor[rgb]{0.56,0.35,0.01}{\textbf{\textit{#1}}}}
\newcommand{\AttributeTok}[1]{\textcolor[rgb]{0.13,0.29,0.53}{#1}}
\newcommand{\BaseNTok}[1]{\textcolor[rgb]{0.00,0.00,0.81}{#1}}
\newcommand{\BuiltInTok}[1]{#1}
\newcommand{\CharTok}[1]{\textcolor[rgb]{0.31,0.60,0.02}{#1}}
\newcommand{\CommentTok}[1]{\textcolor[rgb]{0.56,0.35,0.01}{\textit{#1}}}
\newcommand{\CommentVarTok}[1]{\textcolor[rgb]{0.56,0.35,0.01}{\textbf{\textit{#1}}}}
\newcommand{\ConstantTok}[1]{\textcolor[rgb]{0.56,0.35,0.01}{#1}}
\newcommand{\ControlFlowTok}[1]{\textcolor[rgb]{0.13,0.29,0.53}{\textbf{#1}}}
\newcommand{\DataTypeTok}[1]{\textcolor[rgb]{0.13,0.29,0.53}{#1}}
\newcommand{\DecValTok}[1]{\textcolor[rgb]{0.00,0.00,0.81}{#1}}
\newcommand{\DocumentationTok}[1]{\textcolor[rgb]{0.56,0.35,0.01}{\textbf{\textit{#1}}}}
\newcommand{\ErrorTok}[1]{\textcolor[rgb]{0.64,0.00,0.00}{\textbf{#1}}}
\newcommand{\ExtensionTok}[1]{#1}
\newcommand{\FloatTok}[1]{\textcolor[rgb]{0.00,0.00,0.81}{#1}}
\newcommand{\FunctionTok}[1]{\textcolor[rgb]{0.13,0.29,0.53}{\textbf{#1}}}
\newcommand{\ImportTok}[1]{#1}
\newcommand{\InformationTok}[1]{\textcolor[rgb]{0.56,0.35,0.01}{\textbf{\textit{#1}}}}
\newcommand{\KeywordTok}[1]{\textcolor[rgb]{0.13,0.29,0.53}{\textbf{#1}}}
\newcommand{\NormalTok}[1]{#1}
\newcommand{\OperatorTok}[1]{\textcolor[rgb]{0.81,0.36,0.00}{\textbf{#1}}}
\newcommand{\OtherTok}[1]{\textcolor[rgb]{0.56,0.35,0.01}{#1}}
\newcommand{\PreprocessorTok}[1]{\textcolor[rgb]{0.56,0.35,0.01}{\textit{#1}}}
\newcommand{\RegionMarkerTok}[1]{#1}
\newcommand{\SpecialCharTok}[1]{\textcolor[rgb]{0.81,0.36,0.00}{\textbf{#1}}}
\newcommand{\SpecialStringTok}[1]{\textcolor[rgb]{0.31,0.60,0.02}{#1}}
\newcommand{\StringTok}[1]{\textcolor[rgb]{0.31,0.60,0.02}{#1}}
\newcommand{\VariableTok}[1]{\textcolor[rgb]{0.00,0.00,0.00}{#1}}
\newcommand{\VerbatimStringTok}[1]{\textcolor[rgb]{0.31,0.60,0.02}{#1}}
\newcommand{\WarningTok}[1]{\textcolor[rgb]{0.56,0.35,0.01}{\textbf{\textit{#1}}}}
\usepackage{graphicx}
\makeatletter
\newsavebox\pandoc@box
\newcommand*\pandocbounded[1]{% scales image to fit in text height/width
  \sbox\pandoc@box{#1}%
  \Gscale@div\@tempa{\textheight}{\dimexpr\ht\pandoc@box+\dp\pandoc@box\relax}%
  \Gscale@div\@tempb{\linewidth}{\wd\pandoc@box}%
  \ifdim\@tempb\p@<\@tempa\p@\let\@tempa\@tempb\fi% select the smaller of both
  \ifdim\@tempa\p@<\p@\scalebox{\@tempa}{\usebox\pandoc@box}%
  \else\usebox{\pandoc@box}%
  \fi%
}
% Set default figure placement to htbp
\def\fps@figure{htbp}
\makeatother
\setlength{\emergencystretch}{3em} % prevent overfull lines
\providecommand{\tightlist}{%
  \setlength{\itemsep}{0pt}\setlength{\parskip}{0pt}}
\usepackage{fontspec}
\setmainfont{NanumGothic}
\usepackage{bookmark}
\IfFileExists{xurl.sty}{\usepackage{xurl}}{} % add URL line breaks if available
\urlstyle{same}
\hypersetup{
  pdftitle={Fitting Normal Distribution},
  pdfauthor={coop711},
  hidelinks,
  pdfcreator={LaTeX via pandoc}}

\title{Fitting Normal Distribution}
\author{coop711}
\date{2025-10-08}

\begin{document}
\maketitle

\section{Data}\label{data}

\subsection{From Stigler's}\label{from-stiglers}

\begin{Shaded}
\begin{Highlighting}[]
\NormalTok{knitr}\SpecialCharTok{::}\FunctionTok{include\_graphics}\NormalTok{(}\StringTok{"../pics/quetelet\_soldiers.png"}\NormalTok{)}
\end{Highlighting}
\end{Shaded}

\begin{flushleft}\includegraphics[width=0.5\linewidth]{../pics/quetelet_soldiers} \end{flushleft}

\subsection{Frequency Table}\label{frequency-table}

\begin{itemize}
\tightlist
\item
  케틀레가 작성한 스코틀랜드 군인 5738명의 가슴둘레(인치) 분포표를
  옮기면
\end{itemize}

\begin{Shaded}
\begin{Highlighting}[]
\NormalTok{chest }\OtherTok{\textless{}{-}} \DecValTok{33}\SpecialCharTok{:}\DecValTok{48}
\NormalTok{freq }\OtherTok{\textless{}{-}} \FunctionTok{c}\NormalTok{(}\DecValTok{3}\NormalTok{, }\DecValTok{18}\NormalTok{, }\DecValTok{81}\NormalTok{, }\DecValTok{185}\NormalTok{, }\DecValTok{420}\NormalTok{, }\DecValTok{749}\NormalTok{, }\DecValTok{1073}\NormalTok{, }\DecValTok{1079}\NormalTok{, }\DecValTok{934}\NormalTok{, }\DecValTok{658}\NormalTok{, }\DecValTok{370}\NormalTok{, }\DecValTok{92}\NormalTok{, }\DecValTok{50}\NormalTok{, }\DecValTok{21}\NormalTok{, }\DecValTok{4}\NormalTok{, }\DecValTok{1}\NormalTok{)}
\FunctionTok{data.frame}\NormalTok{(chest, freq)}
\end{Highlighting}
\end{Shaded}

\begin{verbatim}
##    chest freq
## 1     33    3
## 2     34   18
## 3     35   81
## 4     36  185
## 5     37  420
## 6     38  749
## 7     39 1073
## 8     40 1079
## 9     41  934
## 10    42  658
## 11    43  370
## 12    44   92
## 13    45   50
## 14    46   21
## 15    47    4
## 16    48    1
\end{verbatim}

\begin{Shaded}
\begin{Highlighting}[]
\FunctionTok{data.frame}\NormalTok{(}\AttributeTok{Chest =}\NormalTok{ chest, }\AttributeTok{Freq =}\NormalTok{ freq)}
\end{Highlighting}
\end{Shaded}

\begin{verbatim}
##    Chest Freq
## 1     33    3
## 2     34   18
## 3     35   81
## 4     36  185
## 5     37  420
## 6     38  749
## 7     39 1073
## 8     40 1079
## 9     41  934
## 10    42  658
## 11    43  370
## 12    44   92
## 13    45   50
## 14    46   21
## 15    47    4
## 16    48    1
\end{verbatim}

\begin{Shaded}
\begin{Highlighting}[]
\NormalTok{chest\_df }\OtherTok{\textless{}{-}} \FunctionTok{data.frame}\NormalTok{(}\AttributeTok{Chest =}\NormalTok{ chest, }\AttributeTok{Freq =}\NormalTok{ freq)}
\NormalTok{chest\_df}
\end{Highlighting}
\end{Shaded}

\begin{verbatim}
##    Chest Freq
## 1     33    3
## 2     34   18
## 3     35   81
## 4     36  185
## 5     37  420
## 6     38  749
## 7     39 1073
## 8     40 1079
## 9     41  934
## 10    42  658
## 11    43  370
## 12    44   92
## 13    45   50
## 14    46   21
## 15    47    4
## 16    48    1
\end{verbatim}

\begin{Shaded}
\begin{Highlighting}[]
\FunctionTok{str}\NormalTok{(chest\_df)}
\end{Highlighting}
\end{Shaded}

\begin{verbatim}
## 'data.frame':    16 obs. of  2 variables:
##  $ Chest: int  33 34 35 36 37 38 39 40 41 42 ...
##  $ Freq : num  3 18 81 185 420 ...
\end{verbatim}

\subsubsection{Extract Parts of an
Object}\label{extract-parts-of-an-object}

\begin{Shaded}
\begin{Highlighting}[]
\NormalTok{chest\_df}\SpecialCharTok{$}\NormalTok{Freq}
\end{Highlighting}
\end{Shaded}

\begin{verbatim}
##  [1]    3   18   81  185  420  749 1073 1079  934  658  370   92   50   21    4
## [16]    1
\end{verbatim}

\begin{Shaded}
\begin{Highlighting}[]
\NormalTok{chest\_df }\SpecialCharTok{\%\textgreater{}\%}
\NormalTok{  .}\SpecialCharTok{$}\NormalTok{Freq}
\end{Highlighting}
\end{Shaded}

\begin{verbatim}
##  [1]    3   18   81  185  420  749 1073 1079  934  658  370   92   50   21    4
## [16]    1
\end{verbatim}

\begin{Shaded}
\begin{Highlighting}[]
\FunctionTok{str}\NormalTok{(chest\_df}\SpecialCharTok{$}\NormalTok{Freq)}
\end{Highlighting}
\end{Shaded}

\begin{verbatim}
##  num [1:16] 3 18 81 185 420 ...
\end{verbatim}

\begin{Shaded}
\begin{Highlighting}[]
\NormalTok{chest\_df[, }\DecValTok{2}\NormalTok{]}
\end{Highlighting}
\end{Shaded}

\begin{verbatim}
##  [1]    3   18   81  185  420  749 1073 1079  934  658  370   92   50   21    4
## [16]    1
\end{verbatim}

\begin{Shaded}
\begin{Highlighting}[]
\NormalTok{chest\_df }\SpecialCharTok{\%\textgreater{}\%}
  \StringTok{\textasciigrave{}}\AttributeTok{[}\StringTok{\textasciigrave{}}\NormalTok{(, }\DecValTok{2}\NormalTok{)}
\end{Highlighting}
\end{Shaded}

\begin{verbatim}
##  [1]    3   18   81  185  420  749 1073 1079  934  658  370   92   50   21    4
## [16]    1
\end{verbatim}

\begin{Shaded}
\begin{Highlighting}[]
\FunctionTok{str}\NormalTok{(chest\_df[, }\DecValTok{2}\NormalTok{])}
\end{Highlighting}
\end{Shaded}

\begin{verbatim}
##  num [1:16] 3 18 81 185 420 ...
\end{verbatim}

\begin{Shaded}
\begin{Highlighting}[]
\NormalTok{chest\_df[, }\StringTok{"Freq"}\NormalTok{]}
\end{Highlighting}
\end{Shaded}

\begin{verbatim}
##  [1]    3   18   81  185  420  749 1073 1079  934  658  370   92   50   21    4
## [16]    1
\end{verbatim}

\begin{Shaded}
\begin{Highlighting}[]
\NormalTok{chest\_df }\SpecialCharTok{\%\textgreater{}\%}
  \StringTok{\textasciigrave{}}\AttributeTok{[}\StringTok{\textasciigrave{}}\NormalTok{(, }\StringTok{"Freq"}\NormalTok{)}
\end{Highlighting}
\end{Shaded}

\begin{verbatim}
##  [1]    3   18   81  185  420  749 1073 1079  934  658  370   92   50   21    4
## [16]    1
\end{verbatim}

\begin{Shaded}
\begin{Highlighting}[]
\FunctionTok{str}\NormalTok{(chest\_df[, }\StringTok{"Freq"}\NormalTok{])}
\end{Highlighting}
\end{Shaded}

\begin{verbatim}
##  num [1:16] 3 18 81 185 420 ...
\end{verbatim}

\begin{Shaded}
\begin{Highlighting}[]
\NormalTok{chest\_df[}\StringTok{"Freq"}\NormalTok{]}
\end{Highlighting}
\end{Shaded}

\begin{verbatim}
##    Freq
## 1     3
## 2    18
## 3    81
## 4   185
## 5   420
## 6   749
## 7  1073
## 8  1079
## 9   934
## 10  658
## 11  370
## 12   92
## 13   50
## 14   21
## 15    4
## 16    1
\end{verbatim}

\begin{Shaded}
\begin{Highlighting}[]
\NormalTok{chest\_df }\SpecialCharTok{\%\textgreater{}\%}
  \StringTok{\textasciigrave{}}\AttributeTok{[}\StringTok{\textasciigrave{}}\NormalTok{(}\StringTok{"Freq"}\NormalTok{)}
\end{Highlighting}
\end{Shaded}

\begin{verbatim}
##    Freq
## 1     3
## 2    18
## 3    81
## 4   185
## 5   420
## 6   749
## 7  1073
## 8  1079
## 9   934
## 10  658
## 11  370
## 12   92
## 13   50
## 14   21
## 15    4
## 16    1
\end{verbatim}

\begin{Shaded}
\begin{Highlighting}[]
\FunctionTok{str}\NormalTok{(chest\_df[}\StringTok{"Freq"}\NormalTok{])}
\end{Highlighting}
\end{Shaded}

\begin{verbatim}
## 'data.frame':    16 obs. of  1 variable:
##  $ Freq: num  3 18 81 185 420 ...
\end{verbatim}

\begin{Shaded}
\begin{Highlighting}[]
\NormalTok{chest\_df[}\StringTok{"Freq"}\NormalTok{]}\SpecialCharTok{$}\NormalTok{Freq}
\end{Highlighting}
\end{Shaded}

\begin{verbatim}
##  [1]    3   18   81  185  420  749 1073 1079  934  658  370   92   50   21    4
## [16]    1
\end{verbatim}

\begin{Shaded}
\begin{Highlighting}[]
\NormalTok{chest\_df }\SpecialCharTok{\%\textgreater{}\%}
  \StringTok{\textasciigrave{}}\AttributeTok{[}\StringTok{\textasciigrave{}}\NormalTok{(}\StringTok{"Freq"}\NormalTok{) }\SpecialCharTok{\%\textgreater{}\%}
\NormalTok{  .}\SpecialCharTok{$}\NormalTok{Freq}
\end{Highlighting}
\end{Shaded}

\begin{verbatim}
##  [1]    3   18   81  185  420  749 1073 1079  934  658  370   92   50   21    4
## [16]    1
\end{verbatim}

\begin{Shaded}
\begin{Highlighting}[]
\FunctionTok{str}\NormalTok{(chest\_df[}\StringTok{"Freq"}\NormalTok{]}\SpecialCharTok{$}\NormalTok{Freq)}
\end{Highlighting}
\end{Shaded}

\begin{verbatim}
##  num [1:16] 3 18 81 185 420 ...
\end{verbatim}

\begin{Shaded}
\begin{Highlighting}[]
\NormalTok{chest\_df[}\StringTok{"Freq"}\NormalTok{][[}\DecValTok{1}\NormalTok{]]}
\end{Highlighting}
\end{Shaded}

\begin{verbatim}
##  [1]    3   18   81  185  420  749 1073 1079  934  658  370   92   50   21    4
## [16]    1
\end{verbatim}

\begin{Shaded}
\begin{Highlighting}[]
\NormalTok{chest\_df }\SpecialCharTok{\%\textgreater{}\%}
  \StringTok{\textasciigrave{}}\AttributeTok{[}\StringTok{\textasciigrave{}}\NormalTok{(}\StringTok{"Freq"}\NormalTok{) }\SpecialCharTok{\%\textgreater{}\%}
  \StringTok{\textasciigrave{}}\AttributeTok{[[}\StringTok{\textasciigrave{}}\NormalTok{(}\DecValTok{1}\NormalTok{) }
\end{Highlighting}
\end{Shaded}

\begin{verbatim}
##  [1]    3   18   81  185  420  749 1073 1079  934  658  370   92   50   21    4
## [16]    1
\end{verbatim}

\begin{Shaded}
\begin{Highlighting}[]
\CommentTok{\#  \textasciigrave{}[\textasciigrave{}(, 1) }
\CommentTok{\#   \textasciigrave{}[\textasciigrave{}(1)}
\FunctionTok{str}\NormalTok{(chest\_df[}\StringTok{"Freq"}\NormalTok{][[}\DecValTok{1}\NormalTok{]])}
\end{Highlighting}
\end{Shaded}

\begin{verbatim}
##  num [1:16] 3 18 81 185 420 ...
\end{verbatim}

\begin{Shaded}
\begin{Highlighting}[]
\NormalTok{chest\_df[}\DecValTok{2}\NormalTok{]}
\end{Highlighting}
\end{Shaded}

\begin{verbatim}
##    Freq
## 1     3
## 2    18
## 3    81
## 4   185
## 5   420
## 6   749
## 7  1073
## 8  1079
## 9   934
## 10  658
## 11  370
## 12   92
## 13   50
## 14   21
## 15    4
## 16    1
\end{verbatim}

\begin{Shaded}
\begin{Highlighting}[]
\NormalTok{chest\_df }\SpecialCharTok{\%\textgreater{}\%}
  \StringTok{\textasciigrave{}}\AttributeTok{[}\StringTok{\textasciigrave{}}\NormalTok{(}\DecValTok{2}\NormalTok{)}
\end{Highlighting}
\end{Shaded}

\begin{verbatim}
##    Freq
## 1     3
## 2    18
## 3    81
## 4   185
## 5   420
## 6   749
## 7  1073
## 8  1079
## 9   934
## 10  658
## 11  370
## 12   92
## 13   50
## 14   21
## 15    4
## 16    1
\end{verbatim}

\begin{Shaded}
\begin{Highlighting}[]
\FunctionTok{str}\NormalTok{(chest\_df[}\DecValTok{2}\NormalTok{])}
\end{Highlighting}
\end{Shaded}

\begin{verbatim}
## 'data.frame':    16 obs. of  1 variable:
##  $ Freq: num  3 18 81 185 420 ...
\end{verbatim}

\begin{Shaded}
\begin{Highlighting}[]
\NormalTok{chest\_df[}\DecValTok{2}\NormalTok{]}\SpecialCharTok{$}\NormalTok{Freq}
\end{Highlighting}
\end{Shaded}

\begin{verbatim}
##  [1]    3   18   81  185  420  749 1073 1079  934  658  370   92   50   21    4
## [16]    1
\end{verbatim}

\begin{Shaded}
\begin{Highlighting}[]
\NormalTok{chest\_df }\SpecialCharTok{\%\textgreater{}\%}
  \StringTok{\textasciigrave{}}\AttributeTok{[}\StringTok{\textasciigrave{}}\NormalTok{(}\DecValTok{2}\NormalTok{) }\SpecialCharTok{\%\textgreater{}\%}
\NormalTok{  .}\SpecialCharTok{$}\NormalTok{Freq}
\end{Highlighting}
\end{Shaded}

\begin{verbatim}
##  [1]    3   18   81  185  420  749 1073 1079  934  658  370   92   50   21    4
## [16]    1
\end{verbatim}

\begin{Shaded}
\begin{Highlighting}[]
\FunctionTok{str}\NormalTok{(chest\_df[}\DecValTok{2}\NormalTok{]}\SpecialCharTok{$}\NormalTok{Freq)}
\end{Highlighting}
\end{Shaded}

\begin{verbatim}
##  num [1:16] 3 18 81 185 420 ...
\end{verbatim}

\begin{Shaded}
\begin{Highlighting}[]
\NormalTok{chest\_df[}\DecValTok{2}\NormalTok{][[}\DecValTok{1}\NormalTok{]]}
\end{Highlighting}
\end{Shaded}

\begin{verbatim}
##  [1]    3   18   81  185  420  749 1073 1079  934  658  370   92   50   21    4
## [16]    1
\end{verbatim}

\begin{Shaded}
\begin{Highlighting}[]
\NormalTok{chest\_df }\SpecialCharTok{\%\textgreater{}\%}
  \StringTok{\textasciigrave{}}\AttributeTok{[}\StringTok{\textasciigrave{}}\NormalTok{(}\DecValTok{2}\NormalTok{) }\SpecialCharTok{\%\textgreater{}\%}
  \StringTok{\textasciigrave{}}\AttributeTok{[[}\StringTok{\textasciigrave{}}\NormalTok{(}\DecValTok{1}\NormalTok{)}
\end{Highlighting}
\end{Shaded}

\begin{verbatim}
##  [1]    3   18   81  185  420  749 1073 1079  934  658  370   92   50   21    4
## [16]    1
\end{verbatim}

\begin{Shaded}
\begin{Highlighting}[]
\FunctionTok{str}\NormalTok{(chest\_df[}\DecValTok{2}\NormalTok{][[}\DecValTok{1}\NormalTok{]])}
\end{Highlighting}
\end{Shaded}

\begin{verbatim}
##  num [1:16] 3 18 81 185 420 ...
\end{verbatim}

\begin{Shaded}
\begin{Highlighting}[]
\NormalTok{chest\_df[[}\DecValTok{2}\NormalTok{]]}
\end{Highlighting}
\end{Shaded}

\begin{verbatim}
##  [1]    3   18   81  185  420  749 1073 1079  934  658  370   92   50   21    4
## [16]    1
\end{verbatim}

\begin{Shaded}
\begin{Highlighting}[]
\NormalTok{chest\_df }\SpecialCharTok{\%\textgreater{}\%}
  \StringTok{\textasciigrave{}}\AttributeTok{[[}\StringTok{\textasciigrave{}}\NormalTok{(}\DecValTok{2}\NormalTok{)}
\end{Highlighting}
\end{Shaded}

\begin{verbatim}
##  [1]    3   18   81  185  420  749 1073 1079  934  658  370   92   50   21    4
## [16]    1
\end{verbatim}

\begin{Shaded}
\begin{Highlighting}[]
\FunctionTok{str}\NormalTok{(chest\_df[[}\DecValTok{2}\NormalTok{]])}
\end{Highlighting}
\end{Shaded}

\begin{verbatim}
##  num [1:16] 3 18 81 185 420 ...
\end{verbatim}

\begin{itemize}
\tightlist
\item
  33인치인 사람이 3명, 34인치인 사람이 18명 등으로 기록되어 있으나 이는
  구간의 가운데로 이해하여야 함.
\end{itemize}

\subsection{Probability Histogram}\label{probability-histogram}

\begin{itemize}
\tightlist
\item
  \texttt{barplot(height,\ ...)} 은 기본적으로 \texttt{height}만
  주어지면 그릴 수 있음. 확률 히스토그램의 기둥 면적의 합은 1이므로, 각
  기둥의 높이는 각 계급의 돗수를 전체 돗수, 5738명으로 나눠준 값임.
\end{itemize}

\begin{Shaded}
\begin{Highlighting}[]
\NormalTok{total }\OtherTok{\textless{}{-}} \FunctionTok{sum}\NormalTok{(chest\_df}\SpecialCharTok{$}\NormalTok{Freq)}
\FunctionTok{barplot}\NormalTok{(chest\_df}\SpecialCharTok{$}\NormalTok{Freq }\SpecialCharTok{/}\NormalTok{ total)}
\end{Highlighting}
\end{Shaded}

\pandocbounded{\includegraphics[keepaspectratio]{Quetelet_chest_pdf_files/figure-latex/barplot first-1.pdf}}

\begin{Shaded}
\begin{Highlighting}[]
\CommentTok{\#\textgreater{} 위의 두 줄을 \%\textgreater{}\% 로 흘려보내면 \textasciigrave{}total\textasciigrave{}을 만들지 않아도 되는 데 ...}
\NormalTok{chest\_df}\SpecialCharTok{$}\NormalTok{Freq }\SpecialCharTok{\%\textgreater{}\%}
  \StringTok{\textasciigrave{}}\AttributeTok{/}\StringTok{\textasciigrave{}}\NormalTok{(., }\FunctionTok{sum}\NormalTok{(.)) }\SpecialCharTok{\%\textgreater{}\%}
\NormalTok{  barplot}
\NormalTok{chest\_df}\SpecialCharTok{$}\NormalTok{Freq }\SpecialCharTok{\%\textgreater{}\%} 
\NormalTok{  prop.table }\SpecialCharTok{\%\textgreater{}\%}
\CommentTok{\#\textgreater{} R 4.0.0 부터는 proportions 사용 가능}
\CommentTok{\#  proportions \%\textgreater{}\%}
\NormalTok{  barplot}
\CommentTok{\#\textgreater{} 조심! 다음 두 표현은 원하는 그림이 나오지 않음.}
\NormalTok{chest\_df}\SpecialCharTok{$}\NormalTok{Freq }\SpecialCharTok{\%\textgreater{}\%}
  \FunctionTok{barplot}\NormalTok{(. }\SpecialCharTok{/} \FunctionTok{sum}\NormalTok{(.))}
\end{Highlighting}
\end{Shaded}

\pandocbounded{\includegraphics[keepaspectratio]{Quetelet_chest_pdf_files/figure-latex/barplot first-2.pdf}}

\begin{Shaded}
\begin{Highlighting}[]
\NormalTok{chest\_df}\SpecialCharTok{$}\NormalTok{Freq }\SpecialCharTok{\%\textgreater{}\%}
  \FunctionTok{barplot}\NormalTok{(}\StringTok{\textasciigrave{}}\AttributeTok{/}\StringTok{\textasciigrave{}}\NormalTok{(., }\FunctionTok{sum}\NormalTok{(.)))}
\end{Highlighting}
\end{Shaded}

\begin{itemize}
\tightlist
\item
  각 막대의 이름은 계급을 나타내는 가슴둘레 값으로 표현할 수 있고, 막대
  간의 사이를 띄우지 않으며, 디폴트 값으로 주어진 회색 보다는 차라리
  백색이 나으므로 이를 설정해 주면,
\end{itemize}

\begin{Shaded}
\begin{Highlighting}[]
\FunctionTok{barplot}\NormalTok{(chest\_df}\SpecialCharTok{$}\NormalTok{Freq}\SpecialCharTok{/}\NormalTok{total, }
        \AttributeTok{names.arg =} \DecValTok{33}\SpecialCharTok{:}\DecValTok{48}\NormalTok{, }
        \AttributeTok{space =} \DecValTok{0}\NormalTok{, }
        \AttributeTok{col =} \StringTok{"white"}\NormalTok{)}
\end{Highlighting}
\end{Shaded}

\pandocbounded{\includegraphics[keepaspectratio]{Quetelet_chest_pdf_files/figure-latex/barplot white-1.pdf}}

\begin{Shaded}
\begin{Highlighting}[]
\NormalTok{chest\_df}\SpecialCharTok{$}\NormalTok{Freq }\SpecialCharTok{\%\textgreater{}\%}
  \StringTok{\textasciigrave{}}\AttributeTok{/}\StringTok{\textasciigrave{}}\NormalTok{(., }\FunctionTok{sum}\NormalTok{(.)) }\SpecialCharTok{\%\textgreater{}\%}
  \FunctionTok{barplot}\NormalTok{(}\AttributeTok{names.arg =} \DecValTok{33}\SpecialCharTok{:}\DecValTok{48}\NormalTok{, }
          \AttributeTok{space =} \DecValTok{0}\NormalTok{, }
          \AttributeTok{col =} \StringTok{"white"}\NormalTok{)}
\end{Highlighting}
\end{Shaded}

\begin{itemize}
\tightlist
\item
  확률 히스토그램의 정의에 따라 이 막대들의 면적을 합하면 1이 됨에 유의.
\end{itemize}

\subsection{Summary statistics and SD}\label{summary-statistics-and-sd}

\begin{itemize}
\tightlist
\item
  33인치가 3명, 34인치가 18명 등을 한 줄의 긴 벡터로 나타내어야 평균과
  표준편차를 쉽게 계산할 수 있으므로 long format으로 바꾸면,
\end{itemize}

\begin{Shaded}
\begin{Highlighting}[]
\NormalTok{chest\_vec }\OtherTok{\textless{}{-}} \FunctionTok{rep}\NormalTok{(chest\_df}\SpecialCharTok{$}\NormalTok{Chest, chest\_df}\SpecialCharTok{$}\NormalTok{Freq)}
\NormalTok{chest\_vec }\OtherTok{\textless{}{-}}\NormalTok{ chest\_df }\SpecialCharTok{\%$\%}
  \FunctionTok{rep}\NormalTok{(.}\SpecialCharTok{$}\NormalTok{Chest, .}\SpecialCharTok{$}\NormalTok{Freq)}
\FunctionTok{str}\NormalTok{(chest\_vec)}
\end{Highlighting}
\end{Shaded}

\begin{verbatim}
##  int [1:5738] 33 33 33 34 34 34 34 34 34 34 ...
\end{verbatim}

\subsubsection{\texorpdfstring{\texttt{rep()}}{rep()}}\label{rep}

\begin{Shaded}
\begin{Highlighting}[]
\FunctionTok{rep}\NormalTok{(}\DecValTok{1}\SpecialCharTok{:}\DecValTok{3}\NormalTok{, }\AttributeTok{times =} \DecValTok{3}\NormalTok{)}
\end{Highlighting}
\end{Shaded}

\begin{verbatim}
## [1] 1 2 3 1 2 3 1 2 3
\end{verbatim}

\begin{Shaded}
\begin{Highlighting}[]
\FunctionTok{rep}\NormalTok{(}\DecValTok{1}\SpecialCharTok{:}\DecValTok{3}\NormalTok{, }\AttributeTok{each =} \DecValTok{3}\NormalTok{)}
\end{Highlighting}
\end{Shaded}

\begin{verbatim}
## [1] 1 1 1 2 2 2 3 3 3
\end{verbatim}

\begin{Shaded}
\begin{Highlighting}[]
\FunctionTok{rep}\NormalTok{(}\DecValTok{1}\SpecialCharTok{:}\DecValTok{3}\NormalTok{, }\DecValTok{1}\SpecialCharTok{:}\DecValTok{3}\NormalTok{)}
\end{Highlighting}
\end{Shaded}

\begin{verbatim}
## [1] 1 2 2 3 3 3
\end{verbatim}

\begin{itemize}
\tightlist
\item
  \texttt{chest\_vec} 을 이용하여 기초통계와 표준편차를 계산하면,
\end{itemize}

\begin{Shaded}
\begin{Highlighting}[]
\FunctionTok{summary}\NormalTok{(chest\_vec)}
\end{Highlighting}
\end{Shaded}

\begin{verbatim}
##    Min. 1st Qu.  Median    Mean 3rd Qu.    Max. 
##   33.00   38.00   40.00   39.83   41.00   48.00
\end{verbatim}

\begin{Shaded}
\begin{Highlighting}[]
\FunctionTok{sd}\NormalTok{(chest\_vec)}
\end{Highlighting}
\end{Shaded}

\begin{verbatim}
## [1] 2.049616
\end{verbatim}

\subsection{Histogram}\label{histogram}

\begin{itemize}
\tightlist
\item
  히스토그램을 직관적으로 그려보면 \(y\)축은 돗수가 기본값임을 알 수
  있음.
\end{itemize}

\begin{Shaded}
\begin{Highlighting}[]
\FunctionTok{hist}\NormalTok{(chest\_vec)}
\end{Highlighting}
\end{Shaded}

\pandocbounded{\includegraphics[keepaspectratio]{Quetelet_chest_pdf_files/figure-latex/frequency histogram-1.pdf}}

\begin{Shaded}
\begin{Highlighting}[]
\NormalTok{chest\_vec }\SpecialCharTok{\%\textgreater{}\%}
\NormalTok{  hist}
\end{Highlighting}
\end{Shaded}

\pandocbounded{\includegraphics[keepaspectratio]{Quetelet_chest_pdf_files/figure-latex/frequency histogram-2.pdf}}

\begin{itemize}
\tightlist
\item
  정규분포와 비교하기 위해서 \(y\)축을 확률로 나타내려면
\end{itemize}

\begin{Shaded}
\begin{Highlighting}[]
\FunctionTok{hist}\NormalTok{(chest\_vec, }
     \AttributeTok{probability =} \ConstantTok{TRUE}\NormalTok{)}
\end{Highlighting}
\end{Shaded}

\pandocbounded{\includegraphics[keepaspectratio]{Quetelet_chest_pdf_files/figure-latex/probability histogram-1.pdf}}

\begin{Shaded}
\begin{Highlighting}[]
\NormalTok{chest\_vec }\SpecialCharTok{\%\textgreater{}\%}
  \FunctionTok{hist}\NormalTok{(}\AttributeTok{probability =} \ConstantTok{TRUE}\NormalTok{)}
\end{Highlighting}
\end{Shaded}

\pandocbounded{\includegraphics[keepaspectratio]{Quetelet_chest_pdf_files/figure-latex/probability histogram-2.pdf}}

\subsubsection{Inside the histogram}\label{inside-the-histogram}

\begin{itemize}
\tightlist
\item
  실제로 이 히스토그램을 그리는 데 계산된 값들은?
\end{itemize}

\begin{Shaded}
\begin{Highlighting}[]
\NormalTok{(h\_chest }\OtherTok{\textless{}{-}} \FunctionTok{hist}\NormalTok{(chest\_vec, }\AttributeTok{plot =} \ConstantTok{FALSE}\NormalTok{))}
\end{Highlighting}
\end{Shaded}

\begin{verbatim}
## $breaks
##  [1] 33 34 35 36 37 38 39 40 41 42 43 44 45 46 47 48
## 
## $counts
##  [1]   21   81  185  420  749 1073 1079  934  658  370   92   50   21    4    1
## 
## $density
##  [1] 0.0036598118 0.0141164169 0.0322411990 0.0731962356 0.1305332869
##  [6] 0.1869989543 0.1880446148 0.1627744859 0.1146741025 0.0644823980
## [11] 0.0160334611 0.0087138376 0.0036598118 0.0006971070 0.0001742768
## 
## $mids
##  [1] 33.5 34.5 35.5 36.5 37.5 38.5 39.5 40.5 41.5 42.5 43.5 44.5 45.5 46.5 47.5
## 
## $xname
## [1] "chest_vec"
## 
## $equidist
## [1] TRUE
## 
## attr(,"class")
## [1] "histogram"
\end{verbatim}

\begin{Shaded}
\begin{Highlighting}[]
\FunctionTok{list}\NormalTok{(}\AttributeTok{breaks =}\NormalTok{ h\_chest}\SpecialCharTok{$}\NormalTok{breaks, }
     \AttributeTok{counts =}\NormalTok{ h\_chest}\SpecialCharTok{$}\NormalTok{counts, }
     \AttributeTok{density =}\NormalTok{ h\_chest}\SpecialCharTok{$}\NormalTok{density, }
     \AttributeTok{mids =}\NormalTok{ h\_chest}\SpecialCharTok{$}\NormalTok{mids)}
\end{Highlighting}
\end{Shaded}

\begin{verbatim}
## $breaks
##  [1] 33 34 35 36 37 38 39 40 41 42 43 44 45 46 47 48
## 
## $counts
##  [1]   21   81  185  420  749 1073 1079  934  658  370   92   50   21    4    1
## 
## $density
##  [1] 0.0036598118 0.0141164169 0.0322411990 0.0731962356 0.1305332869
##  [6] 0.1869989543 0.1880446148 0.1627744859 0.1146741025 0.0644823980
## [11] 0.0160334611 0.0087138376 0.0036598118 0.0006971070 0.0001742768
## 
## $mids
##  [1] 33.5 34.5 35.5 36.5 37.5 38.5 39.5 40.5 41.5 42.5 43.5 44.5 45.5 46.5 47.5
\end{verbatim}

\begin{Shaded}
\begin{Highlighting}[]
\NormalTok{chest\_vec }\SpecialCharTok{\%\textgreater{}\%}
  \FunctionTok{hist}\NormalTok{(}\AttributeTok{plot =} \ConstantTok{FALSE}\NormalTok{) }\SpecialCharTok{\%\textgreater{}\%}
  \FunctionTok{list}\NormalTok{(}\AttributeTok{breaks =}\NormalTok{ .}\SpecialCharTok{$}\NormalTok{breaks, }
       \AttributeTok{counts =}\NormalTok{ .}\SpecialCharTok{$}\NormalTok{counts, }
       \AttributeTok{density =}\NormalTok{ .}\SpecialCharTok{$}\NormalTok{density, }
       \AttributeTok{mids =}\NormalTok{ .}\SpecialCharTok{$}\NormalTok{mids)}
\end{Highlighting}
\end{Shaded}

\begin{verbatim}
## [[1]]
## $breaks
##  [1] 33 34 35 36 37 38 39 40 41 42 43 44 45 46 47 48
## 
## $counts
##  [1]   21   81  185  420  749 1073 1079  934  658  370   92   50   21    4    1
## 
## $density
##  [1] 0.0036598118 0.0141164169 0.0322411990 0.0731962356 0.1305332869
##  [6] 0.1869989543 0.1880446148 0.1627744859 0.1146741025 0.0644823980
## [11] 0.0160334611 0.0087138376 0.0036598118 0.0006971070 0.0001742768
## 
## $mids
##  [1] 33.5 34.5 35.5 36.5 37.5 38.5 39.5 40.5 41.5 42.5 43.5 44.5 45.5 46.5 47.5
## 
## $xname
## [1] "."
## 
## $equidist
## [1] TRUE
## 
## attr(,"class")
## [1] "histogram"
## 
## $breaks
##  [1] 33 34 35 36 37 38 39 40 41 42 43 44 45 46 47 48
## 
## $counts
##  [1]   21   81  185  420  749 1073 1079  934  658  370   92   50   21    4    1
## 
## $density
##  [1] 0.0036598118 0.0141164169 0.0322411990 0.0731962356 0.1305332869
##  [6] 0.1869989543 0.1880446148 0.1627744859 0.1146741025 0.0644823980
## [11] 0.0160334611 0.0087138376 0.0036598118 0.0006971070 0.0001742768
## 
## $mids
##  [1] 33.5 34.5 35.5 36.5 37.5 38.5 39.5 40.5 41.5 42.5 43.5 44.5 45.5 46.5 47.5
\end{verbatim}

\begin{itemize}
\tightlist
\item
  평균값과 표준편차로부터 히스토그램의 위치가 0.5만큼 왼쪽으로 치우쳐
  있다는 것을 알 수 있음. 제자리에 옮겨 놓기 위해서 \texttt{breaks}
  매개변수를 32.5부터 48.5까지 1간격으로 설정
\end{itemize}

\begin{Shaded}
\begin{Highlighting}[]
\FunctionTok{hist}\NormalTok{(chest\_vec, }
     \AttributeTok{probability =} \ConstantTok{TRUE}\NormalTok{, }
     \AttributeTok{breaks =} \FloatTok{32.5}\SpecialCharTok{:}\FloatTok{48.5}\NormalTok{)}
\end{Highlighting}
\end{Shaded}

\pandocbounded{\includegraphics[keepaspectratio]{Quetelet_chest_pdf_files/figure-latex/move 0.5 inches-1.pdf}}

\begin{itemize}
\tightlist
\item
  위의 히스토그램을 그리느라고 계산된 값들은?
\end{itemize}

\begin{Shaded}
\begin{Highlighting}[]
\NormalTok{h\_chest\_2 }\OtherTok{\textless{}{-}} \FunctionTok{hist}\NormalTok{(chest\_vec, }
                  \AttributeTok{breaks =} \FloatTok{32.5}\SpecialCharTok{:}\FloatTok{48.5}\NormalTok{, }
                  \AttributeTok{plot =} \ConstantTok{FALSE}\NormalTok{)}
\FunctionTok{list}\NormalTok{(}\AttributeTok{breaks =}\NormalTok{ h\_chest\_2}\SpecialCharTok{$}\NormalTok{breaks, }
     \AttributeTok{counts =}\NormalTok{ h\_chest\_2}\SpecialCharTok{$}\NormalTok{counts, }
     \AttributeTok{density =}\NormalTok{ h\_chest\_2}\SpecialCharTok{$}\NormalTok{density, }
     \AttributeTok{mids =}\NormalTok{ h\_chest\_2}\SpecialCharTok{$}\NormalTok{mids)}
\end{Highlighting}
\end{Shaded}

\begin{verbatim}
## $breaks
##  [1] 32.5 33.5 34.5 35.5 36.5 37.5 38.5 39.5 40.5 41.5 42.5 43.5 44.5 45.5 46.5
## [16] 47.5 48.5
## 
## $counts
##  [1]    3   18   81  185  420  749 1073 1079  934  658  370   92   50   21    4
## [16]    1
## 
## $density
##  [1] 0.0005228303 0.0031369815 0.0141164169 0.0322411990 0.0731962356
##  [6] 0.1305332869 0.1869989543 0.1880446148 0.1627744859 0.1146741025
## [11] 0.0644823980 0.0160334611 0.0087138376 0.0036598118 0.0006971070
## [16] 0.0001742768
## 
## $mids
##  [1] 33 34 35 36 37 38 39 40 41 42 43 44 45 46 47 48
\end{verbatim}

\begin{Shaded}
\begin{Highlighting}[]
\NormalTok{chest\_vec }\SpecialCharTok{\%\textgreater{}\%}
  \FunctionTok{hist}\NormalTok{(}\AttributeTok{breaks =} \FloatTok{32.5}\SpecialCharTok{:}\FloatTok{48.5}\NormalTok{,}
       \AttributeTok{plot =} \ConstantTok{FALSE}\NormalTok{) }\SpecialCharTok{\%\textgreater{}\%}
  \FunctionTok{list}\NormalTok{(}\AttributeTok{breaks =}\NormalTok{ .}\SpecialCharTok{$}\NormalTok{breaks, }
       \AttributeTok{counts =}\NormalTok{ .}\SpecialCharTok{$}\NormalTok{counts, }
       \AttributeTok{density =}\NormalTok{ .}\SpecialCharTok{$}\NormalTok{density, }
       \AttributeTok{mids =}\NormalTok{ .}\SpecialCharTok{$}\NormalTok{mids)}
\end{Highlighting}
\end{Shaded}

\begin{verbatim}
## [[1]]
## $breaks
##  [1] 32.5 33.5 34.5 35.5 36.5 37.5 38.5 39.5 40.5 41.5 42.5 43.5 44.5 45.5 46.5
## [16] 47.5 48.5
## 
## $counts
##  [1]    3   18   81  185  420  749 1073 1079  934  658  370   92   50   21    4
## [16]    1
## 
## $density
##  [1] 0.0005228303 0.0031369815 0.0141164169 0.0322411990 0.0731962356
##  [6] 0.1305332869 0.1869989543 0.1880446148 0.1627744859 0.1146741025
## [11] 0.0644823980 0.0160334611 0.0087138376 0.0036598118 0.0006971070
## [16] 0.0001742768
## 
## $mids
##  [1] 33 34 35 36 37 38 39 40 41 42 43 44 45 46 47 48
## 
## $xname
## [1] "."
## 
## $equidist
## [1] TRUE
## 
## attr(,"class")
## [1] "histogram"
## 
## $breaks
##  [1] 32.5 33.5 34.5 35.5 36.5 37.5 38.5 39.5 40.5 41.5 42.5 43.5 44.5 45.5 46.5
## [16] 47.5 48.5
## 
## $counts
##  [1]    3   18   81  185  420  749 1073 1079  934  658  370   92   50   21    4
## [16]    1
## 
## $density
##  [1] 0.0005228303 0.0031369815 0.0141164169 0.0322411990 0.0731962356
##  [6] 0.1305332869 0.1869989543 0.1880446148 0.1627744859 0.1146741025
## [11] 0.0644823980 0.0160334611 0.0087138376 0.0036598118 0.0006971070
## [16] 0.0001742768
## 
## $mids
##  [1] 33 34 35 36 37 38 39 40 41 42 43 44 45 46 47 48
\end{verbatim}

\begin{itemize}
\tightlist
\item
  히스토그램을 보기 쉽게 하기 위해서 메인 타이틀과 서브 타이틀, x축
  라벨, y축 라벨 설정
\end{itemize}

\begin{Shaded}
\begin{Highlighting}[]
\NormalTok{main\_title }\OtherTok{\textless{}{-}} \StringTok{"Fitting Normal Distribution"}
\CommentTok{\# sub\_title \textless{}{-} "Chest Circumferences of Scottish Soldiers"}
\NormalTok{sub\_title }\OtherTok{\textless{}{-}} \StringTok{""}
\NormalTok{x\_lab }\OtherTok{\textless{}{-}} \StringTok{"Chest Circumferences (inches)"}
\NormalTok{y\_lab }\OtherTok{\textless{}{-}} \StringTok{"Proportion"}
\FunctionTok{hist}\NormalTok{(chest\_vec, }
     \AttributeTok{breaks =} \FloatTok{32.5}\SpecialCharTok{:}\FloatTok{48.5}\NormalTok{, }
     \AttributeTok{probability =} \ConstantTok{TRUE}\NormalTok{, }
     \AttributeTok{main =}\NormalTok{ main\_title, }
     \AttributeTok{sub =}\NormalTok{ sub\_title, }
     \AttributeTok{xlab =}\NormalTok{ x\_lab, }
     \AttributeTok{ylab =}\NormalTok{ y\_lab)}
\end{Highlighting}
\end{Shaded}

\pandocbounded{\includegraphics[keepaspectratio]{Quetelet_chest_pdf_files/figure-latex/annotations-1.pdf}}

\subsection{\texorpdfstring{Mean \(\pm\) SD contains 2/3 of total number
of
counts}{Mean \textbackslash pm SD contains 2/3 of total number of counts}}\label{mean-pm-sd-contains-23-of-total-number-of-counts}

\begin{itemize}
\tightlist
\item
  평균을 중심으로 \(\pm\)표준편차 만큼 떨어진 자료를 붉은 색
  수직점선으로 표시.
\end{itemize}

\begin{Shaded}
\begin{Highlighting}[]
\NormalTok{mean\_chest }\OtherTok{\textless{}{-}} \FunctionTok{mean}\NormalTok{(chest\_vec)}
\NormalTok{sd\_chest }\OtherTok{\textless{}{-}} \FunctionTok{sd}\NormalTok{(chest\_vec)}
\NormalTok{x\_lower }\OtherTok{\textless{}{-}}\NormalTok{ mean\_chest }\SpecialCharTok{{-}}\NormalTok{ sd\_chest}
\NormalTok{x\_upper }\OtherTok{\textless{}{-}}\NormalTok{ mean\_chest }\SpecialCharTok{+}\NormalTok{ sd\_chest}
\NormalTok{sd\_chest }\OtherTok{\textless{}{-}}\NormalTok{ chest\_vec }\SpecialCharTok{\%\textgreater{}\%}
\NormalTok{  sd}
\NormalTok{x\_lower }\OtherTok{\textless{}{-}}\NormalTok{ chest\_vec }\SpecialCharTok{\%\textgreater{}\%}
\NormalTok{  mean }\SpecialCharTok{\%\textgreater{}\%}
  \StringTok{\textasciigrave{}}\AttributeTok{{-}}\StringTok{\textasciigrave{}}\NormalTok{(sd\_chest)}
\NormalTok{x\_upper }\OtherTok{\textless{}{-}}\NormalTok{ chest\_vec }\SpecialCharTok{\%\textgreater{}\%}
\NormalTok{  mean }\SpecialCharTok{\%\textgreater{}\%}
  \StringTok{\textasciigrave{}}\AttributeTok{+}\StringTok{\textasciigrave{}}\NormalTok{(sd\_chest)}
\FunctionTok{hist}\NormalTok{(chest\_vec, }
     \AttributeTok{breaks =} \FloatTok{32.5}\SpecialCharTok{:}\FloatTok{48.5}\NormalTok{, }
     \AttributeTok{probability =} \ConstantTok{TRUE}\NormalTok{, }
     \AttributeTok{main =}\NormalTok{ main\_title, }
     \AttributeTok{sub =}\NormalTok{ sub\_title, }
     \AttributeTok{xlab =}\NormalTok{ x\_lab, }
     \AttributeTok{ylab =}\NormalTok{ y\_lab)}
\FunctionTok{abline}\NormalTok{(}\AttributeTok{v =} \FunctionTok{c}\NormalTok{(x\_lower, x\_upper), }
       \AttributeTok{lty =} \DecValTok{2}\NormalTok{, }
       \AttributeTok{col =} \StringTok{"red"}\NormalTok{)}
\end{Highlighting}
\end{Shaded}

\pandocbounded{\includegraphics[keepaspectratio]{Quetelet_chest_pdf_files/figure-latex/mean and sd-1.pdf}}

\begin{itemize}
\tightlist
\item
  그 사이의 영역을 빗금으로 표시하기 위하여 다각형의 좌표를 계산
\end{itemize}

\begin{Shaded}
\begin{Highlighting}[]
\NormalTok{h\_chest\_2}\SpecialCharTok{$}\NormalTok{density[}\DecValTok{6}\SpecialCharTok{:}\DecValTok{10}\NormalTok{]}
\end{Highlighting}
\end{Shaded}

\begin{verbatim}
## [1] 0.1305333 0.1869990 0.1880446 0.1627745 0.1146741
\end{verbatim}

\begin{Shaded}
\begin{Highlighting}[]
\NormalTok{y }\OtherTok{\textless{}{-}}\NormalTok{ h\_chest\_2}\SpecialCharTok{$}\NormalTok{density[}\DecValTok{6}\SpecialCharTok{:}\DecValTok{10}\NormalTok{]}
\end{Highlighting}
\end{Shaded}

\begin{itemize}
\tightlist
\item
  5개의 직사각형으로 파악하고 향후 면적 계산을 쉽게 하기 위하여 다음과
  같이 좌표 설정
\end{itemize}

\begin{Shaded}
\begin{Highlighting}[]
\NormalTok{x\_coord }\OtherTok{\textless{}{-}} \FunctionTok{rep}\NormalTok{(}\FunctionTok{c}\NormalTok{(x\_lower, }\FloatTok{38.5}\SpecialCharTok{:}\FloatTok{41.5}\NormalTok{, x\_upper), }\AttributeTok{each =} \DecValTok{2}\NormalTok{)}
\NormalTok{y\_coord }\OtherTok{\textless{}{-}} \FunctionTok{c}\NormalTok{(}\DecValTok{0}\NormalTok{, }\FunctionTok{rep}\NormalTok{(y, }\AttributeTok{each =} \DecValTok{2}\NormalTok{), }\DecValTok{0}\NormalTok{)}
\NormalTok{poly\_df }\OtherTok{\textless{}{-}} \FunctionTok{data.frame}\NormalTok{(}\AttributeTok{x =}\NormalTok{ x\_coord, }\AttributeTok{y =}\NormalTok{ y\_coord)}
\FunctionTok{hist}\NormalTok{(chest\_vec, }
     \AttributeTok{breaks =} \FloatTok{32.5}\SpecialCharTok{:}\FloatTok{48.5}\NormalTok{, }
     \AttributeTok{probability =} \ConstantTok{TRUE}\NormalTok{, }
     \AttributeTok{main =}\NormalTok{ main\_title, }
     \AttributeTok{sub =}\NormalTok{ sub\_title, }
     \AttributeTok{xlab =}\NormalTok{ x\_lab, }
     \AttributeTok{ylab =}\NormalTok{ y\_lab)}
\FunctionTok{abline}\NormalTok{(}\AttributeTok{v =} \FunctionTok{c}\NormalTok{(x\_lower, x\_upper), }
       \AttributeTok{lty =} \DecValTok{2}\NormalTok{, }
       \AttributeTok{col =} \StringTok{"red"}\NormalTok{)}
\CommentTok{\# polygon(x\_coord, y\_coord, density = 20)}
\FunctionTok{polygon}\NormalTok{(poly\_df, }
\CommentTok{\#        col = "grey",}
\CommentTok{\#        border = NA)}
        \AttributeTok{density =} \DecValTok{20}\NormalTok{)}
\end{Highlighting}
\end{Shaded}

\pandocbounded{\includegraphics[keepaspectratio]{Quetelet_chest_pdf_files/figure-latex/5 rectangles-1.pdf}}

\begin{itemize}
\tightlist
\item
  이론적으로 빗금친 부분의 면적은
  \texttt{pnorm(1)\ -\ pnorm(-1)\ =}0.6826895에 가까울 것으로 예상. 5개
  직사각형의 면적을 구하여 합하는 과정은 다음과 같음.
\end{itemize}

\begin{Shaded}
\begin{Highlighting}[]
\FunctionTok{options}\NormalTok{(}\AttributeTok{digits =} \DecValTok{3}\NormalTok{)}
\NormalTok{x\_area }\OtherTok{\textless{}{-}} \FunctionTok{c}\NormalTok{(x\_lower, }\FloatTok{38.5}\SpecialCharTok{:}\FloatTok{41.5}\NormalTok{, x\_upper)}
\NormalTok{y}
\end{Highlighting}
\end{Shaded}

\begin{verbatim}
## [1] 0.131 0.187 0.188 0.163 0.115
\end{verbatim}

\begin{Shaded}
\begin{Highlighting}[]
\FunctionTok{diff}\NormalTok{(x\_area)}
\end{Highlighting}
\end{Shaded}

\begin{verbatim}
## [1] 0.718 1.000 1.000 1.000 0.381
\end{verbatim}

\begin{Shaded}
\begin{Highlighting}[]
\FunctionTok{diff}\NormalTok{(x\_area) }\SpecialCharTok{*}\NormalTok{ y}
\end{Highlighting}
\end{Shaded}

\begin{verbatim}
## [1] 0.0937 0.1870 0.1880 0.1628 0.0437
\end{verbatim}

\begin{Shaded}
\begin{Highlighting}[]
\FunctionTok{sum}\NormalTok{(}\FunctionTok{diff}\NormalTok{(x\_area) }\SpecialCharTok{*}\NormalTok{ y)}
\end{Highlighting}
\end{Shaded}

\begin{verbatim}
## [1] 0.675
\end{verbatim}

\begin{Shaded}
\begin{Highlighting}[]
\FunctionTok{source}\NormalTok{(}\StringTok{"./area.R"}\NormalTok{)}
\NormalTok{area\_R}
\end{Highlighting}
\end{Shaded}

\begin{verbatim}
## function (x, y) 
## {
##     sum(diff(x) * (head(y, -1) + tail(y, -1))/2)
## }
\end{verbatim}

\begin{Shaded}
\begin{Highlighting}[]
\FunctionTok{area\_R}\NormalTok{(x\_coord, y\_coord)}
\end{Highlighting}
\end{Shaded}

\begin{verbatim}
## [1] 0.675
\end{verbatim}

\subsection{Comparison with normal
curve}\label{comparison-with-normal-curve}

\begin{itemize}
\tightlist
\item
  이론적인 정규분포 밀도함수 곡선을 히스토그램에 덧붙여 그림.
\end{itemize}

\begin{Shaded}
\begin{Highlighting}[]
\NormalTok{x\_chest }\OtherTok{\textless{}{-}} \FunctionTok{seq}\NormalTok{(}\FloatTok{32.5}\NormalTok{, }\FloatTok{48.5}\NormalTok{, }
               \AttributeTok{length =} \DecValTok{1000}\NormalTok{)}
\NormalTok{y\_norm }\OtherTok{\textless{}{-}} \FunctionTok{dnorm}\NormalTok{(x\_chest, }
                \AttributeTok{mean =}\NormalTok{ mean\_chest, }
                \AttributeTok{sd =}\NormalTok{ sd\_chest)}
\FunctionTok{hist}\NormalTok{(chest\_vec, }
     \AttributeTok{breaks =} \FloatTok{32.5}\SpecialCharTok{:}\FloatTok{48.5}\NormalTok{, }
     \AttributeTok{probability =} \ConstantTok{TRUE}\NormalTok{, }
     \AttributeTok{main =}\NormalTok{ main\_title, }
     \AttributeTok{sub =}\NormalTok{ sub\_title, }
     \AttributeTok{xlab =}\NormalTok{ x\_lab, }
     \AttributeTok{ylab =}\NormalTok{ y\_lab)}
\FunctionTok{abline}\NormalTok{(}\AttributeTok{v =} \FunctionTok{c}\NormalTok{(x\_lower, x\_upper), }
       \AttributeTok{lty =} \DecValTok{2}\NormalTok{, }
       \AttributeTok{col =} \StringTok{"red"}\NormalTok{)}
\CommentTok{\# abline(v = c(38, 42), lty = 2, col = "red")}
\FunctionTok{polygon}\NormalTok{(poly\_df, }
        \AttributeTok{density =} \DecValTok{20}\NormalTok{)}
\CommentTok{\# polygon(x\_coord, y\_coord, density = 20)}
\FunctionTok{lines}\NormalTok{(x\_chest, y\_norm, }\AttributeTok{col =} \StringTok{"red"}\NormalTok{)}
\end{Highlighting}
\end{Shaded}

\pandocbounded{\includegraphics[keepaspectratio]{Quetelet_chest_pdf_files/figure-latex/normal curve added-1.pdf}}

\subsection{Changing tick marks of x
axis}\label{changing-tick-marks-of-x-axis}

\begin{itemize}
\tightlist
\item
  default로 주어지는 \(x\)축의 눈금을 제대로 볼 수 있게 고치려면,
\end{itemize}

\begin{Shaded}
\begin{Highlighting}[]
\FunctionTok{hist}\NormalTok{(chest\_vec, }
     \AttributeTok{breaks =} \FloatTok{32.5}\SpecialCharTok{:}\FloatTok{48.5}\NormalTok{, }
     \AttributeTok{probability =} \ConstantTok{TRUE}\NormalTok{, }
     \AttributeTok{main =}\NormalTok{ main\_title, }
     \AttributeTok{sub =}\NormalTok{ sub\_title, }
     \AttributeTok{xlab =}\NormalTok{ x\_lab, }
     \AttributeTok{ylab =}\NormalTok{ y\_lab, }
     \AttributeTok{axes =} \ConstantTok{FALSE}\NormalTok{)}
\FunctionTok{abline}\NormalTok{(}\AttributeTok{v =} \FunctionTok{c}\NormalTok{(x\_lower, x\_upper), }
       \AttributeTok{lty =} \DecValTok{2}\NormalTok{, }
       \AttributeTok{col =} \StringTok{"red"}\NormalTok{)}
\FunctionTok{polygon}\NormalTok{(poly\_df, }
        \AttributeTok{density =} \DecValTok{20}\NormalTok{)}
\CommentTok{\# polygon(x\_coord, y\_coord, density = 20)}
\FunctionTok{lines}\NormalTok{(x\_chest, y\_norm, }\AttributeTok{col =} \StringTok{"red"}\NormalTok{)}
\FunctionTok{axis}\NormalTok{(}\AttributeTok{side =} \DecValTok{1}\NormalTok{, }
     \AttributeTok{at =} \FunctionTok{seq}\NormalTok{(}\DecValTok{32}\NormalTok{, }\DecValTok{48}\NormalTok{, }\AttributeTok{by =} \DecValTok{2}\NormalTok{), }
     \AttributeTok{labels =} \FunctionTok{seq}\NormalTok{(}\DecValTok{32}\NormalTok{, }\DecValTok{48}\NormalTok{, }\AttributeTok{by =} \DecValTok{2}\NormalTok{))}
\FunctionTok{axis}\NormalTok{(}\AttributeTok{side =} \DecValTok{2}\NormalTok{)}
\end{Highlighting}
\end{Shaded}

\pandocbounded{\includegraphics[keepaspectratio]{Quetelet_chest_pdf_files/figure-latex/x axis-1.pdf}}

\subsection{ggplot}\label{ggplot}

\begin{itemize}
\tightlist
\item
  data frame으로 작업.
\end{itemize}

\subsubsection{Basic histogram}\label{basic-histogram}

\begin{Shaded}
\begin{Highlighting}[]
\FunctionTok{library}\NormalTok{(ggplot2)}
\CommentTok{\# theme\_update(plot.title = element\_text(hjust = 0.5)) }
\NormalTok{g0 }\OtherTok{\textless{}{-}} \FunctionTok{ggplot}\NormalTok{(}\AttributeTok{data =} \FunctionTok{data.frame}\NormalTok{(chest\_vec), }
             \AttributeTok{mapping =} \FunctionTok{aes}\NormalTok{(}\AttributeTok{x =}\NormalTok{ chest\_vec)) }
\NormalTok{(g1 }\OtherTok{\textless{}{-}}\NormalTok{ g0 }\SpecialCharTok{+}
    \FunctionTok{stat\_bin}\NormalTok{(}\FunctionTok{aes}\NormalTok{(}\AttributeTok{y =} \FunctionTok{after\_stat}\NormalTok{(density)),}
\CommentTok{\#   stat\_bin(aes(y = ..density..), \#\textgreater{} ggplot 새 버전에서 퇴출된 코드}
             \AttributeTok{binwidth =} \DecValTok{1}\NormalTok{, }
             \AttributeTok{fill =} \StringTok{"white"}\NormalTok{, }
             \AttributeTok{colour =} \StringTok{"black"}\NormalTok{))}
\end{Highlighting}
\end{Shaded}

\pandocbounded{\includegraphics[keepaspectratio]{Quetelet_chest_pdf_files/figure-latex/ggplots-1.pdf}}

\begin{Shaded}
\begin{Highlighting}[]
\CommentTok{\#\textgreater{} 다음 방법들도 동일한 결과 }
\CommentTok{\# (g1 \textless{}{-} g0 + }
\CommentTok{\#   stat\_count(fill = "white", }
\CommentTok{\#              colour = "black"))}
\CommentTok{\# (g1 \textless{}{-} g0 + }
\CommentTok{\#    geom\_histogram(aes(y = ..density..), }
\CommentTok{\#                   binwidth = 1, }
\CommentTok{\#                   fill = "white", }
\CommentTok{\#                   colour = "black"))}
\CommentTok{\# (g1 \textless{}{-} g0 + }
\CommentTok{\#  geom\_histogram(aes(y = ..density..), }
\CommentTok{\#                binwidth = 1, }
\CommentTok{\#                breaks = 32.5:48.5, }
\CommentTok{\#                fill = "white", }
\CommentTok{\#                colour = "black"))}
\end{Highlighting}
\end{Shaded}

\subsubsection{\texorpdfstring{Mean \(\pm\)
SD}{Mean \textbackslash pm SD}}\label{mean-pm-sd}

\begin{Shaded}
\begin{Highlighting}[]
\NormalTok{(g2 }\OtherTok{\textless{}{-}}\NormalTok{ g1 }\SpecialCharTok{+} 
   \FunctionTok{geom\_vline}\NormalTok{(}\AttributeTok{xintercept =} \FunctionTok{c}\NormalTok{(x\_lower, x\_upper), }
              \AttributeTok{linetype =} \StringTok{"dotted"}\NormalTok{, }
              \AttributeTok{colour =} \StringTok{"red"}\NormalTok{)) }
\end{Highlighting}
\end{Shaded}

\pandocbounded{\includegraphics[keepaspectratio]{Quetelet_chest_pdf_files/figure-latex/mean plus minus sd-1.pdf}}

\subsubsection{x-axis label and main
title}\label{x-axis-label-and-main-title}

\begin{Shaded}
\begin{Highlighting}[]
\NormalTok{(g3 }\OtherTok{\textless{}{-}}\NormalTok{ g2 }\SpecialCharTok{+} 
   \FunctionTok{theme\_bw}\NormalTok{() }\SpecialCharTok{+} 
\CommentTok{\#   xlab(x\_lab) + }
\CommentTok{\#   ylab(y\_lab) +}
\CommentTok{\#   ggtitle(main\_title) +}
\CommentTok{\#\textgreater{} 위의 세 코드를 한 번에 }
   \FunctionTok{labs}\NormalTok{(}\AttributeTok{title =}\NormalTok{ main\_title,}
        \AttributeTok{x =}\NormalTok{ x\_lab, }
        \AttributeTok{y =}\NormalTok{ y\_lab) }\SpecialCharTok{+} 
   \FunctionTok{theme}\NormalTok{(}\AttributeTok{plot.title =} \FunctionTok{element\_text}\NormalTok{(}\AttributeTok{hjust =} \FloatTok{0.5}\NormalTok{,}
                                   \AttributeTok{size =} \DecValTok{20}\NormalTok{)))}
\end{Highlighting}
\end{Shaded}

\pandocbounded{\includegraphics[keepaspectratio]{Quetelet_chest_pdf_files/figure-latex/xlab and ggtitle-1.pdf}}

\subsubsection{Shading the area}\label{shading-the-area}

\begin{Shaded}
\begin{Highlighting}[]
\NormalTok{(g4 }\OtherTok{\textless{}{-}}\NormalTok{ g3 }\SpecialCharTok{+} 
   \FunctionTok{geom\_polygon}\NormalTok{(}\AttributeTok{data =}\NormalTok{ poly\_df, }
                \AttributeTok{mapping =} \FunctionTok{aes}\NormalTok{(}\AttributeTok{x =}\NormalTok{ x, }\AttributeTok{y =}\NormalTok{ y), }
                \AttributeTok{alpha =} \FloatTok{0.5}\NormalTok{, }
                \AttributeTok{fill =} \StringTok{"grey"}\NormalTok{))}
\end{Highlighting}
\end{Shaded}

\pandocbounded{\includegraphics[keepaspectratio]{Quetelet_chest_pdf_files/figure-latex/polygon-1.pdf}}

\subsubsection{Normal curve added}\label{normal-curve-added}

\begin{Shaded}
\begin{Highlighting}[]
\CommentTok{\# x\_curve \textless{}{-} seq(32.5, 48.5, length = 100)}
\CommentTok{\# y\_curve \textless{}{-} dnorm(x\_curve, mean = mean\_chest, sd = sd\_chest)}
\NormalTok{curve\_df }\OtherTok{\textless{}{-}} \FunctionTok{data.frame}\NormalTok{(}\AttributeTok{x =}\NormalTok{ x\_chest, }\AttributeTok{y =}\NormalTok{ y\_norm)}
\NormalTok{(g5 }\OtherTok{\textless{}{-}}\NormalTok{ g4 }\SpecialCharTok{+} 
  \FunctionTok{geom\_line}\NormalTok{(}\AttributeTok{data =}\NormalTok{ curve\_df, }
            \AttributeTok{mapping =} \FunctionTok{aes}\NormalTok{(}\AttributeTok{x =}\NormalTok{ x, }\AttributeTok{y =}\NormalTok{ y), }
            \AttributeTok{colour =} \StringTok{"blue"}\NormalTok{))}
\end{Highlighting}
\end{Shaded}

\pandocbounded{\includegraphics[keepaspectratio]{Quetelet_chest_pdf_files/figure-latex/normal curve-1.pdf}}

\subsubsection{x-axis tick marks}\label{x-axis-tick-marks}

\begin{Shaded}
\begin{Highlighting}[]
\NormalTok{(g6 }\OtherTok{\textless{}{-}}\NormalTok{ g5 }\SpecialCharTok{+} 
   \FunctionTok{scale\_x\_continuous}\NormalTok{(}\AttributeTok{breaks =} \FunctionTok{seq}\NormalTok{(}\DecValTok{32}\NormalTok{, }\DecValTok{48}\NormalTok{, }\AttributeTok{by =} \DecValTok{2}\NormalTok{), }
                      \AttributeTok{labels =} \FunctionTok{seq}\NormalTok{(}\DecValTok{32}\NormalTok{, }\DecValTok{48}\NormalTok{, }\AttributeTok{by =} \DecValTok{2}\NormalTok{)))}
\end{Highlighting}
\end{Shaded}

\pandocbounded{\includegraphics[keepaspectratio]{Quetelet_chest_pdf_files/figure-latex/tick marks-1.pdf}}

\subsection{Comments}\label{comments}

R 코드에 대하여 학습한 내용

케틀레가 스코틀랜드 군인 5738명의 가슴둘레 자료를 정규분포에 적합시킨
이야기

\end{document}
