% Options for packages loaded elsewhere
\PassOptionsToPackage{unicode}{hyperref}
\PassOptionsToPackage{hyphens}{url}
\documentclass[
]{article}
\usepackage{xcolor}
\usepackage[margin=1in]{geometry}
\usepackage{amsmath,amssymb}
\setcounter{secnumdepth}{-\maxdimen} % remove section numbering
\usepackage{iftex}
\ifPDFTeX
  \usepackage[T1]{fontenc}
  \usepackage[utf8]{inputenc}
  \usepackage{textcomp} % provide euro and other symbols
\else % if luatex or xetex
  \usepackage{unicode-math} % this also loads fontspec
  \defaultfontfeatures{Scale=MatchLowercase}
  \defaultfontfeatures[\rmfamily]{Ligatures=TeX,Scale=1}
\fi
\usepackage{lmodern}
\ifPDFTeX\else
  % xetex/luatex font selection
\fi
% Use upquote if available, for straight quotes in verbatim environments
\IfFileExists{upquote.sty}{\usepackage{upquote}}{}
\IfFileExists{microtype.sty}{% use microtype if available
  \usepackage[]{microtype}
  \UseMicrotypeSet[protrusion]{basicmath} % disable protrusion for tt fonts
}{}
\makeatletter
\@ifundefined{KOMAClassName}{% if non-KOMA class
  \IfFileExists{parskip.sty}{%
    \usepackage{parskip}
  }{% else
    \setlength{\parindent}{0pt}
    \setlength{\parskip}{6pt plus 2pt minus 1pt}}
}{% if KOMA class
  \KOMAoptions{parskip=half}}
\makeatother
\usepackage{color}
\usepackage{fancyvrb}
\newcommand{\VerbBar}{|}
\newcommand{\VERB}{\Verb[commandchars=\\\{\}]}
\DefineVerbatimEnvironment{Highlighting}{Verbatim}{commandchars=\\\{\}}
% Add ',fontsize=\small' for more characters per line
\usepackage{framed}
\definecolor{shadecolor}{RGB}{248,248,248}
\newenvironment{Shaded}{\begin{snugshade}}{\end{snugshade}}
\newcommand{\AlertTok}[1]{\textcolor[rgb]{0.94,0.16,0.16}{#1}}
\newcommand{\AnnotationTok}[1]{\textcolor[rgb]{0.56,0.35,0.01}{\textbf{\textit{#1}}}}
\newcommand{\AttributeTok}[1]{\textcolor[rgb]{0.13,0.29,0.53}{#1}}
\newcommand{\BaseNTok}[1]{\textcolor[rgb]{0.00,0.00,0.81}{#1}}
\newcommand{\BuiltInTok}[1]{#1}
\newcommand{\CharTok}[1]{\textcolor[rgb]{0.31,0.60,0.02}{#1}}
\newcommand{\CommentTok}[1]{\textcolor[rgb]{0.56,0.35,0.01}{\textit{#1}}}
\newcommand{\CommentVarTok}[1]{\textcolor[rgb]{0.56,0.35,0.01}{\textbf{\textit{#1}}}}
\newcommand{\ConstantTok}[1]{\textcolor[rgb]{0.56,0.35,0.01}{#1}}
\newcommand{\ControlFlowTok}[1]{\textcolor[rgb]{0.13,0.29,0.53}{\textbf{#1}}}
\newcommand{\DataTypeTok}[1]{\textcolor[rgb]{0.13,0.29,0.53}{#1}}
\newcommand{\DecValTok}[1]{\textcolor[rgb]{0.00,0.00,0.81}{#1}}
\newcommand{\DocumentationTok}[1]{\textcolor[rgb]{0.56,0.35,0.01}{\textbf{\textit{#1}}}}
\newcommand{\ErrorTok}[1]{\textcolor[rgb]{0.64,0.00,0.00}{\textbf{#1}}}
\newcommand{\ExtensionTok}[1]{#1}
\newcommand{\FloatTok}[1]{\textcolor[rgb]{0.00,0.00,0.81}{#1}}
\newcommand{\FunctionTok}[1]{\textcolor[rgb]{0.13,0.29,0.53}{\textbf{#1}}}
\newcommand{\ImportTok}[1]{#1}
\newcommand{\InformationTok}[1]{\textcolor[rgb]{0.56,0.35,0.01}{\textbf{\textit{#1}}}}
\newcommand{\KeywordTok}[1]{\textcolor[rgb]{0.13,0.29,0.53}{\textbf{#1}}}
\newcommand{\NormalTok}[1]{#1}
\newcommand{\OperatorTok}[1]{\textcolor[rgb]{0.81,0.36,0.00}{\textbf{#1}}}
\newcommand{\OtherTok}[1]{\textcolor[rgb]{0.56,0.35,0.01}{#1}}
\newcommand{\PreprocessorTok}[1]{\textcolor[rgb]{0.56,0.35,0.01}{\textit{#1}}}
\newcommand{\RegionMarkerTok}[1]{#1}
\newcommand{\SpecialCharTok}[1]{\textcolor[rgb]{0.81,0.36,0.00}{\textbf{#1}}}
\newcommand{\SpecialStringTok}[1]{\textcolor[rgb]{0.31,0.60,0.02}{#1}}
\newcommand{\StringTok}[1]{\textcolor[rgb]{0.31,0.60,0.02}{#1}}
\newcommand{\VariableTok}[1]{\textcolor[rgb]{0.00,0.00,0.00}{#1}}
\newcommand{\VerbatimStringTok}[1]{\textcolor[rgb]{0.31,0.60,0.02}{#1}}
\newcommand{\WarningTok}[1]{\textcolor[rgb]{0.56,0.35,0.01}{\textbf{\textit{#1}}}}
\usepackage{graphicx}
\makeatletter
\newsavebox\pandoc@box
\newcommand*\pandocbounded[1]{% scales image to fit in text height/width
  \sbox\pandoc@box{#1}%
  \Gscale@div\@tempa{\textheight}{\dimexpr\ht\pandoc@box+\dp\pandoc@box\relax}%
  \Gscale@div\@tempb{\linewidth}{\wd\pandoc@box}%
  \ifdim\@tempb\p@<\@tempa\p@\let\@tempa\@tempb\fi% select the smaller of both
  \ifdim\@tempa\p@<\p@\scalebox{\@tempa}{\usebox\pandoc@box}%
  \else\usebox{\pandoc@box}%
  \fi%
}
% Set default figure placement to htbp
\def\fps@figure{htbp}
\makeatother
\setlength{\emergencystretch}{3em} % prevent overfull lines
\providecommand{\tightlist}{%
  \setlength{\itemsep}{0pt}\setlength{\parskip}{0pt}}
\usepackage{fontspec}
\setmainfont{NanumGothic}
\usepackage{bookmark}
\IfFileExists{xurl.sty}{\usepackage{xurl}}{} % add URL line breaks if available
\urlstyle{same}
\hypersetup{
  pdftitle={Crimtab Data for Simulation of T-Distribution},
  pdfauthor={coop711},
  hidelinks,
  pdfcreator={LaTeX via pandoc}}

\title{Crimtab Data for Simulation of T-Distribution}
\author{coop711}
\date{2025-10-08}

\begin{document}
\maketitle

\subsection{Data Loading}\label{data-loading}

\begin{Shaded}
\begin{Highlighting}[]
\FunctionTok{load}\NormalTok{(}\StringTok{"./crimtab.RData"}\NormalTok{)}
\FunctionTok{ls}\NormalTok{()}
\end{Highlighting}
\end{Shaded}

\begin{verbatim}
## [1] "crimtab_2"       "crimtab_df"      "crimtab_long"    "crimtab_long_df"
\end{verbatim}

\begin{Shaded}
\begin{Highlighting}[]
\FunctionTok{ls.str}\NormalTok{()}
\end{Highlighting}
\end{Shaded}

\begin{verbatim}
## crimtab_2 :  'table' int [1:42, 1:22] 0 0 0 0 0 0 1 0 0 0 ...
## crimtab_df : 'data.frame':   924 obs. of  3 variables:
##  $ finger: num  9.4 9.5 9.6 9.7 9.8 9.9 10 10.1 10.2 10.3 ...
##  $ height: num  56 56 56 56 56 56 56 56 56 56 ...
##  $ Freq  : int  0 0 0 0 0 0 1 0 0 0 ...
## crimtab_long :  num [1:3000, 1:2] 10 10.3 9.9 10.2 10.2 10.3 10.4 10.7 10 10.1 ...
## crimtab_long_df : 'data.frame':  3000 obs. of  2 variables:
##  $ finger: num  10 10.3 9.9 10.2 10.2 10.3 10.4 10.7 10 10.1 ...
##  $ height: num  56 57 58 58 58 58 58 58 59 59 ...
\end{verbatim}

\begin{Shaded}
\begin{Highlighting}[]
\FunctionTok{head}\NormalTok{(crimtab\_long\_df, }
     \AttributeTok{n =} \DecValTok{20}\NormalTok{)}
\end{Highlighting}
\end{Shaded}

\begin{verbatim}
##    finger height
## 1    10.0     56
## 2    10.3     57
## 3     9.9     58
## 4    10.2     58
## 5    10.2     58
## 6    10.3     58
## 7    10.4     58
## 8    10.7     58
## 9    10.0     59
## 10   10.1     59
## 11   10.2     59
## 12   10.2     59
## 13   10.3     59
## 14   10.3     59
## 15   10.3     59
## 16   10.4     59
## 17   10.5     59
## 18   10.6     59
## 19   10.7     59
## 20   10.7     59
\end{verbatim}

\subsection{Student 의 Simulation
재현}\label{student-uxc758-simulation-uxc7acuxd604}

\subsubsection{Sample t-values}\label{sample-t-values}

3,000장의 카드를 잘 섞는 것은 \texttt{sample()} 이용.

\begin{Shaded}
\begin{Highlighting}[]
\CommentTok{\# set.seed(113)}
\NormalTok{crimtab\_shuffle }\OtherTok{\textless{}{-}}\NormalTok{ crimtab\_long\_df[}\FunctionTok{sample}\NormalTok{(}\DecValTok{1}\SpecialCharTok{:}\DecValTok{3000}\NormalTok{), ]}
\FunctionTok{head}\NormalTok{(crimtab\_shuffle, }
     \AttributeTok{n =} \DecValTok{10}\NormalTok{)}
\end{Highlighting}
\end{Shaded}

\begin{verbatim}
##      finger height
## 51     10.7     60
## 2230   11.9     67
## 2574   12.1     68
## 2423   11.5     68
## 2035   11.3     67
## 1118   11.0     65
## 1201   11.3     65
## 2494   11.7     68
## 1145   11.2     65
## 757    11.0     64
\end{verbatim}

각 표본의 크기가 4인 750개의 표본을 만드는 작업은 \texttt{rep()} 이용.

\begin{Shaded}
\begin{Highlighting}[]
\NormalTok{sample\_id }\OtherTok{\textless{}{-}} \FunctionTok{as.factor}\NormalTok{(}\FunctionTok{rep}\NormalTok{(}\DecValTok{1}\SpecialCharTok{:}\DecValTok{750}\NormalTok{, }\AttributeTok{each =} \DecValTok{4}\NormalTok{))}
\FunctionTok{head}\NormalTok{(sample\_id, }\AttributeTok{n =} \DecValTok{10}\NormalTok{)}
\end{Highlighting}
\end{Shaded}

\begin{verbatim}
##  [1] 1 1 1 1 2 2 2 2 3 3
## 750 Levels: 1 2 3 4 5 6 7 8 9 10 11 12 13 14 15 16 17 18 19 20 21 22 23 ... 750
\end{verbatim}

각 표본의 평균과 표준편차 계산에는 \texttt{tapply()} 이용.

\begin{Shaded}
\begin{Highlighting}[]
\NormalTok{finger\_sample\_mean }\OtherTok{\textless{}{-}} \FunctionTok{tapply}\NormalTok{(crimtab\_shuffle[, }\StringTok{"finger"}\NormalTok{], }
                             \AttributeTok{INDEX =}\NormalTok{ sample\_id, }
                             \AttributeTok{FUN =}\NormalTok{ mean)}
\NormalTok{finger\_sample\_sd }\OtherTok{\textless{}{-}} \FunctionTok{tapply}\NormalTok{(crimtab\_shuffle[, }\StringTok{"finger"}\NormalTok{], }
                           \AttributeTok{INDEX =}\NormalTok{ sample\_id, }
                           \AttributeTok{FUN =}\NormalTok{ sd)}
\FunctionTok{str}\NormalTok{(finger\_sample\_mean)}
\end{Highlighting}
\end{Shaded}

\begin{verbatim}
##  num [1:750(1d)] 11.6 11.3 11.1 11.4 11.6 ...
##  - attr(*, "dimnames")=List of 1
##   ..$ : chr [1:750] "1" "2" "3" "4" ...
\end{verbatim}

\begin{Shaded}
\begin{Highlighting}[]
\FunctionTok{str}\NormalTok{(finger\_sample\_sd)}
\end{Highlighting}
\end{Shaded}

\begin{verbatim}
##  num [1:750(1d)] 0.619 0.287 0.15 0.538 0.85 ...
##  - attr(*, "dimnames")=List of 1
##   ..$ : chr [1:750] "1" "2" "3" "4" ...
\end{verbatim}

t-통계량 계산. Student는 표준편차 계산에서 분모에 \(n\)을 사용하고
히스토그램을 그려 비교하였으나 자유도 3인 t-분포와 비교하기 위하여
\(t=\frac{\bar{X_n}-\mu}{\hat{SD}/\sqrt{n}}\)을 계산함. (여기서
\(\hat{SD}\)는 표본 표준편차)

\begin{Shaded}
\begin{Highlighting}[]
\NormalTok{sample\_t }\OtherTok{\textless{}{-}}\NormalTok{ (finger\_sample\_mean }\SpecialCharTok{{-}} \FunctionTok{mean}\NormalTok{(crimtab\_long\_df[, }\StringTok{"finger"}\NormalTok{])) }\SpecialCharTok{/}\NormalTok{ (finger\_sample\_sd}\SpecialCharTok{/}\FunctionTok{sqrt}\NormalTok{(}\DecValTok{4}\NormalTok{))}
\FunctionTok{str}\NormalTok{(sample\_t)}
\end{Highlighting}
\end{Shaded}

\begin{verbatim}
##  num [1:750(1d)] 0.00851 -1.54836 -5.63156 -0.64108 0.18267 ...
##  - attr(*, "dimnames")=List of 1
##   ..$ : chr [1:750] "1" "2" "3" "4" ...
\end{verbatim}

계산한 t-통계량 값들의 평균과 표준편차, 히스토그램을 그리고 자유도 3인
t-분포의 밀도함수 및 표준정규곡선과 비교. 우선 모두 같은 값들이 나와서
분모가 0인 경우가 있는지 파악. 있으면 모평균과 비교하여 양수인 경우 +6,
음수인 경우 -6 값 부여(Student가 한 일)

\begin{Shaded}
\begin{Highlighting}[]
\NormalTok{t\_inf }\OtherTok{\textless{}{-}} \FunctionTok{is.infinite}\NormalTok{(sample\_t)}
\NormalTok{sample\_t[t\_inf]}
\end{Highlighting}
\end{Shaded}

\begin{verbatim}
## named numeric(0)
\end{verbatim}

\begin{Shaded}
\begin{Highlighting}[]
\NormalTok{sample\_t[t\_inf] }\OtherTok{\textless{}{-}} \DecValTok{6} \SpecialCharTok{*} \FunctionTok{sign}\NormalTok{(sample\_t[t\_inf])}
\end{Highlighting}
\end{Shaded}

문제되는 값이 없는 것을 확인하고, 평균과 표준편차 계산. 자유도 \(n\)인
t-분포의 평균과 표준편차는 각각 0과 \(\sqrt{\frac{n}{n-2}}\)임을 상기할
것. -6이나 +6보다 큰 값이 상당히 자주 나온다는 점에 유의.

\begin{Shaded}
\begin{Highlighting}[]
\FunctionTok{mean}\NormalTok{(sample\_t)}
\end{Highlighting}
\end{Shaded}

\begin{verbatim}
## [1] -0.04779039
\end{verbatim}

\begin{Shaded}
\begin{Highlighting}[]
\FunctionTok{sd}\NormalTok{(sample\_t)}
\end{Highlighting}
\end{Shaded}

\begin{verbatim}
## [1] 1.6343
\end{verbatim}

\begin{Shaded}
\begin{Highlighting}[]
\FunctionTok{summary}\NormalTok{(sample\_t)}
\end{Highlighting}
\end{Shaded}

\begin{verbatim}
##      Min.   1st Qu.    Median      Mean   3rd Qu.      Max. 
## -8.822952 -0.850822  0.008208 -0.047790  0.816831  6.764549
\end{verbatim}

\subsubsection{Histogram and Density
Curves}\label{histogram-and-density-curves}

t-통계량들의 히스토그램을 그리고, 자유도 3인 t의 밀도함수, 표준정규분포
밀도함수와 비교.

\begin{Shaded}
\begin{Highlighting}[]
\CommentTok{\# hist(sample\_t, prob = TRUE, ylim = c(0, 0.5))}
\CommentTok{\# hist(sample\_t, prob = TRUE, nclass = 20, xlim = c({-}6, 6), ylim = c(0, 0.5), main = "Histogram of Sample t{-}statistics", xlab = "Sampled t{-}values")}
\CommentTok{\# hist(sample\_t, prob = TRUE, nclass = 50, xlim = c({-}6, 6), ylim = c(0, 0.5), main = "Histogram of Sample t{-}statistics", xlab = "Sampled t{-}values")}
\FunctionTok{hist}\NormalTok{(sample\_t, }
     \AttributeTok{prob =} \ConstantTok{TRUE}\NormalTok{, }
     \AttributeTok{breaks =} \FunctionTok{seq}\NormalTok{(}\SpecialCharTok{{-}}\DecValTok{20}\NormalTok{, }\DecValTok{20}\NormalTok{, }\AttributeTok{by =} \FloatTok{0.5}\NormalTok{), }
     \AttributeTok{xlim =} \FunctionTok{c}\NormalTok{(}\SpecialCharTok{{-}}\DecValTok{6}\NormalTok{, }\DecValTok{6}\NormalTok{), }
     \AttributeTok{ylim =} \FunctionTok{c}\NormalTok{(}\DecValTok{0}\NormalTok{, }\FloatTok{0.5}\NormalTok{), }
     \AttributeTok{main =} \StringTok{"Histogram of Sample t{-}statistics"}\NormalTok{, }
     \AttributeTok{xlab =} \StringTok{"Sampled t{-}values"}\NormalTok{)}
\FunctionTok{lines}\NormalTok{(}\FunctionTok{seq}\NormalTok{(}\SpecialCharTok{{-}}\DecValTok{6}\NormalTok{, }\DecValTok{6}\NormalTok{, }\AttributeTok{by =} \FloatTok{0.01}\NormalTok{), }
      \FunctionTok{dt}\NormalTok{(}\FunctionTok{seq}\NormalTok{(}\SpecialCharTok{{-}}\DecValTok{6}\NormalTok{, }\DecValTok{6}\NormalTok{, }\AttributeTok{by =} \FloatTok{0.01}\NormalTok{), }\AttributeTok{df =} \DecValTok{3}\NormalTok{), }
      \AttributeTok{col =} \StringTok{"blue"}\NormalTok{)}
\FunctionTok{lines}\NormalTok{(}\FunctionTok{seq}\NormalTok{(}\SpecialCharTok{{-}}\DecValTok{6}\NormalTok{, }\DecValTok{6}\NormalTok{, }\AttributeTok{by =} \FloatTok{0.01}\NormalTok{), }
      \FunctionTok{dnorm}\NormalTok{(}\FunctionTok{seq}\NormalTok{(}\SpecialCharTok{{-}}\DecValTok{6}\NormalTok{, }\DecValTok{6}\NormalTok{, }\AttributeTok{by =} \FloatTok{0.01}\NormalTok{)), }
      \AttributeTok{col =} \StringTok{"red"}\NormalTok{)}
\FunctionTok{legend}\NormalTok{(}\StringTok{"topright"}\NormalTok{, }
       \AttributeTok{inset =} \FloatTok{0.05}\NormalTok{, }
       \AttributeTok{legend =} \FunctionTok{c}\NormalTok{(}\StringTok{"t with df = 3"}\NormalTok{, }\StringTok{"standard normal"}\NormalTok{),}
       \AttributeTok{lty =} \DecValTok{1}\NormalTok{, }
       \AttributeTok{col =} \FunctionTok{c}\NormalTok{(}\StringTok{"blue"}\NormalTok{, }\StringTok{"red"}\NormalTok{))}
\end{Highlighting}
\end{Shaded}

\pandocbounded{\includegraphics[keepaspectratio]{crimtab_student_simulation_pdf_files/figure-latex/comparison-1.pdf}}

\subsubsection{ggplot}\label{ggplot}

\begin{Shaded}
\begin{Highlighting}[]
\FunctionTok{library}\NormalTok{(ggplot2)}

\CommentTok{\# 샘플 데이터로 히스토그램 생성}
\FunctionTok{ggplot}\NormalTok{(}\FunctionTok{data.frame}\NormalTok{(}\AttributeTok{x =}\NormalTok{ sample\_t), }\FunctionTok{aes}\NormalTok{(}\AttributeTok{x =}\NormalTok{ x)) }\SpecialCharTok{+}
  \FunctionTok{geom\_histogram}\NormalTok{(}\FunctionTok{aes}\NormalTok{(}\AttributeTok{y =} \FunctionTok{after\_stat}\NormalTok{(density)), }
                 \AttributeTok{breaks =} \FunctionTok{seq}\NormalTok{(}\SpecialCharTok{{-}}\DecValTok{20}\NormalTok{, }\DecValTok{20}\NormalTok{, }\AttributeTok{by =} \FloatTok{0.5}\NormalTok{), }
                 \AttributeTok{color =} \StringTok{"black"}\NormalTok{, }\AttributeTok{fill =} \StringTok{"white"}\NormalTok{) }\SpecialCharTok{+}
  \FunctionTok{xlim}\NormalTok{(}\FunctionTok{c}\NormalTok{(}\SpecialCharTok{{-}}\DecValTok{6}\NormalTok{, }\DecValTok{6}\NormalTok{)) }\SpecialCharTok{+} 
  \FunctionTok{ylim}\NormalTok{(}\FunctionTok{c}\NormalTok{(}\DecValTok{0}\NormalTok{, }\FloatTok{0.5}\NormalTok{)) }\SpecialCharTok{+}
  \FunctionTok{stat\_function}\NormalTok{(}\AttributeTok{fun =}\NormalTok{ dt, }\AttributeTok{args =} \FunctionTok{list}\NormalTok{(}\AttributeTok{df =} \DecValTok{3}\NormalTok{), }\FunctionTok{aes}\NormalTok{(}\AttributeTok{color =} \StringTok{"t with df = 3"}\NormalTok{), }\AttributeTok{linetype =} \StringTok{"solid"}\NormalTok{) }\SpecialCharTok{+}
  \FunctionTok{stat\_function}\NormalTok{(}\AttributeTok{fun =}\NormalTok{ dnorm, }\FunctionTok{aes}\NormalTok{(}\AttributeTok{color =} \StringTok{"standard normal"}\NormalTok{), }\AttributeTok{linetype =} \StringTok{"solid"}\NormalTok{) }\SpecialCharTok{+}
  \FunctionTok{labs}\NormalTok{(}\AttributeTok{title =} \StringTok{"Histogram of Sample t{-}statistics"}\NormalTok{, }
       \AttributeTok{x =} \StringTok{"Sampled t{-}values"}\NormalTok{, }
       \AttributeTok{color =} \ConstantTok{NULL}\NormalTok{) }\SpecialCharTok{+}  \CommentTok{\# 범례 제목을 빈 문자열로 설정}
  \FunctionTok{theme\_minimal}\NormalTok{() }\SpecialCharTok{+}
  \FunctionTok{theme}\NormalTok{(}\AttributeTok{plot.title =} \FunctionTok{element\_text}\NormalTok{(}\AttributeTok{hjust =} \FloatTok{0.5}\NormalTok{)) }\SpecialCharTok{+}
  \FunctionTok{scale\_color\_manual}\NormalTok{(}\AttributeTok{values =} \FunctionTok{c}\NormalTok{(}\StringTok{"t with df = 3"} \OtherTok{=} \StringTok{"blue"}\NormalTok{, }\StringTok{"standard normal"} \OtherTok{=} \StringTok{"red"}\NormalTok{)) }\SpecialCharTok{+}
  \FunctionTok{guides}\NormalTok{(}\AttributeTok{color =} \FunctionTok{guide\_legend}\NormalTok{(}\AttributeTok{override.aes =} \FunctionTok{list}\NormalTok{(}\AttributeTok{linetype =} \FunctionTok{c}\NormalTok{(}\StringTok{"solid"}\NormalTok{, }\StringTok{"solid"}\NormalTok{), }\AttributeTok{size =} \FloatTok{1.5}\NormalTok{))) }\SpecialCharTok{+}
  \FunctionTok{theme}\NormalTok{(}\AttributeTok{legend.position =} \StringTok{"inside"}\NormalTok{, }
        \AttributeTok{legend.justification =} \FunctionTok{c}\NormalTok{(}\FloatTok{0.8}\NormalTok{, }\FloatTok{0.8}\NormalTok{), }\CommentTok{\# 범례의 위치를 플롯 내부로 지정}
        \AttributeTok{legend.background =} \FunctionTok{element\_rect}\NormalTok{(}\AttributeTok{fill =} \StringTok{"white"}\NormalTok{, }\AttributeTok{color =} \StringTok{"black"}\NormalTok{))}
\end{Highlighting}
\end{Shaded}

\pandocbounded{\includegraphics[keepaspectratio]{crimtab_student_simulation_pdf_files/figure-latex/ggplot-1.pdf}}

\subsubsection{QQnorm}\label{qqnorm}

\texttt{qqnorm()} 을 그려보면 정규분포와 꼬리에서 큰 차이가 난다는 것을
알 수 있음.

\begin{Shaded}
\begin{Highlighting}[]
\FunctionTok{qqnorm}\NormalTok{(sample\_t)}
\end{Highlighting}
\end{Shaded}

\pandocbounded{\includegraphics[keepaspectratio]{crimtab_student_simulation_pdf_files/figure-latex/qqnorm-1.pdf}}

\end{document}
