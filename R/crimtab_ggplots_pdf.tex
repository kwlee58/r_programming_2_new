% Options for packages loaded elsewhere
\PassOptionsToPackage{unicode}{hyperref}
\PassOptionsToPackage{hyphens}{url}
\documentclass[
]{article}
\usepackage{xcolor}
\usepackage[margin=1in]{geometry}
\usepackage{amsmath,amssymb}
\setcounter{secnumdepth}{-\maxdimen} % remove section numbering
\usepackage{iftex}
\ifPDFTeX
  \usepackage[T1]{fontenc}
  \usepackage[utf8]{inputenc}
  \usepackage{textcomp} % provide euro and other symbols
\else % if luatex or xetex
  \usepackage{unicode-math} % this also loads fontspec
  \defaultfontfeatures{Scale=MatchLowercase}
  \defaultfontfeatures[\rmfamily]{Ligatures=TeX,Scale=1}
\fi
\usepackage{lmodern}
\ifPDFTeX\else
  % xetex/luatex font selection
\fi
% Use upquote if available, for straight quotes in verbatim environments
\IfFileExists{upquote.sty}{\usepackage{upquote}}{}
\IfFileExists{microtype.sty}{% use microtype if available
  \usepackage[]{microtype}
  \UseMicrotypeSet[protrusion]{basicmath} % disable protrusion for tt fonts
}{}
\makeatletter
\@ifundefined{KOMAClassName}{% if non-KOMA class
  \IfFileExists{parskip.sty}{%
    \usepackage{parskip}
  }{% else
    \setlength{\parindent}{0pt}
    \setlength{\parskip}{6pt plus 2pt minus 1pt}}
}{% if KOMA class
  \KOMAoptions{parskip=half}}
\makeatother
\usepackage{color}
\usepackage{fancyvrb}
\newcommand{\VerbBar}{|}
\newcommand{\VERB}{\Verb[commandchars=\\\{\}]}
\DefineVerbatimEnvironment{Highlighting}{Verbatim}{commandchars=\\\{\}}
% Add ',fontsize=\small' for more characters per line
\usepackage{framed}
\definecolor{shadecolor}{RGB}{248,248,248}
\newenvironment{Shaded}{\begin{snugshade}}{\end{snugshade}}
\newcommand{\AlertTok}[1]{\textcolor[rgb]{0.94,0.16,0.16}{#1}}
\newcommand{\AnnotationTok}[1]{\textcolor[rgb]{0.56,0.35,0.01}{\textbf{\textit{#1}}}}
\newcommand{\AttributeTok}[1]{\textcolor[rgb]{0.13,0.29,0.53}{#1}}
\newcommand{\BaseNTok}[1]{\textcolor[rgb]{0.00,0.00,0.81}{#1}}
\newcommand{\BuiltInTok}[1]{#1}
\newcommand{\CharTok}[1]{\textcolor[rgb]{0.31,0.60,0.02}{#1}}
\newcommand{\CommentTok}[1]{\textcolor[rgb]{0.56,0.35,0.01}{\textit{#1}}}
\newcommand{\CommentVarTok}[1]{\textcolor[rgb]{0.56,0.35,0.01}{\textbf{\textit{#1}}}}
\newcommand{\ConstantTok}[1]{\textcolor[rgb]{0.56,0.35,0.01}{#1}}
\newcommand{\ControlFlowTok}[1]{\textcolor[rgb]{0.13,0.29,0.53}{\textbf{#1}}}
\newcommand{\DataTypeTok}[1]{\textcolor[rgb]{0.13,0.29,0.53}{#1}}
\newcommand{\DecValTok}[1]{\textcolor[rgb]{0.00,0.00,0.81}{#1}}
\newcommand{\DocumentationTok}[1]{\textcolor[rgb]{0.56,0.35,0.01}{\textbf{\textit{#1}}}}
\newcommand{\ErrorTok}[1]{\textcolor[rgb]{0.64,0.00,0.00}{\textbf{#1}}}
\newcommand{\ExtensionTok}[1]{#1}
\newcommand{\FloatTok}[1]{\textcolor[rgb]{0.00,0.00,0.81}{#1}}
\newcommand{\FunctionTok}[1]{\textcolor[rgb]{0.13,0.29,0.53}{\textbf{#1}}}
\newcommand{\ImportTok}[1]{#1}
\newcommand{\InformationTok}[1]{\textcolor[rgb]{0.56,0.35,0.01}{\textbf{\textit{#1}}}}
\newcommand{\KeywordTok}[1]{\textcolor[rgb]{0.13,0.29,0.53}{\textbf{#1}}}
\newcommand{\NormalTok}[1]{#1}
\newcommand{\OperatorTok}[1]{\textcolor[rgb]{0.81,0.36,0.00}{\textbf{#1}}}
\newcommand{\OtherTok}[1]{\textcolor[rgb]{0.56,0.35,0.01}{#1}}
\newcommand{\PreprocessorTok}[1]{\textcolor[rgb]{0.56,0.35,0.01}{\textit{#1}}}
\newcommand{\RegionMarkerTok}[1]{#1}
\newcommand{\SpecialCharTok}[1]{\textcolor[rgb]{0.81,0.36,0.00}{\textbf{#1}}}
\newcommand{\SpecialStringTok}[1]{\textcolor[rgb]{0.31,0.60,0.02}{#1}}
\newcommand{\StringTok}[1]{\textcolor[rgb]{0.31,0.60,0.02}{#1}}
\newcommand{\VariableTok}[1]{\textcolor[rgb]{0.00,0.00,0.00}{#1}}
\newcommand{\VerbatimStringTok}[1]{\textcolor[rgb]{0.31,0.60,0.02}{#1}}
\newcommand{\WarningTok}[1]{\textcolor[rgb]{0.56,0.35,0.01}{\textbf{\textit{#1}}}}
\usepackage{graphicx}
\makeatletter
\newsavebox\pandoc@box
\newcommand*\pandocbounded[1]{% scales image to fit in text height/width
  \sbox\pandoc@box{#1}%
  \Gscale@div\@tempa{\textheight}{\dimexpr\ht\pandoc@box+\dp\pandoc@box\relax}%
  \Gscale@div\@tempb{\linewidth}{\wd\pandoc@box}%
  \ifdim\@tempb\p@<\@tempa\p@\let\@tempa\@tempb\fi% select the smaller of both
  \ifdim\@tempa\p@<\p@\scalebox{\@tempa}{\usebox\pandoc@box}%
  \else\usebox{\pandoc@box}%
  \fi%
}
% Set default figure placement to htbp
\def\fps@figure{htbp}
\makeatother
\setlength{\emergencystretch}{3em} % prevent overfull lines
\providecommand{\tightlist}{%
  \setlength{\itemsep}{0pt}\setlength{\parskip}{0pt}}
\usepackage{fontspec}
\setmainfont{NanumGothic}
\usepackage{bookmark}
\IfFileExists{xurl.sty}{\usepackage{xurl}}{} % add URL line breaks if available
\urlstyle{same}
\hypersetup{
  pdftitle={Student 3000 Criminal Data : ggplot},
  pdfauthor={coop711},
  hidelinks,
  pdfcreator={LaTeX via pandoc}}

\title{Student 3000 Criminal Data : ggplot}
\author{coop711}
\date{2025-10-08}

\begin{document}
\maketitle

\subsection{Working Data Loading}\label{working-data-loading}

\begin{Shaded}
\begin{Highlighting}[]
\FunctionTok{library}\NormalTok{(magrittr)}
\FunctionTok{load}\NormalTok{(}\StringTok{"./crimtab.RData"}\NormalTok{)}
\FunctionTok{ls}\NormalTok{()}
\end{Highlighting}
\end{Shaded}

\begin{verbatim}
## [1] "crimtab_2"       "crimtab_df"      "crimtab_long"    "crimtab_long_df"
\end{verbatim}

\begin{Shaded}
\begin{Highlighting}[]
\FunctionTok{ls.str}\NormalTok{()}
\end{Highlighting}
\end{Shaded}

\begin{verbatim}
## crimtab_2 :  'table' int [1:42, 1:22] 0 0 0 0 0 0 1 0 0 0 ...
## crimtab_df : 'data.frame':   924 obs. of  3 variables:
##  $ finger: num  9.4 9.5 9.6 9.7 9.8 9.9 10 10.1 10.2 10.3 ...
##  $ height: num  56 56 56 56 56 56 56 56 56 56 ...
##  $ Freq  : int  0 0 0 0 0 0 1 0 0 0 ...
## crimtab_long :  num [1:3000, 1:2] 10 10.3 9.9 10.2 10.2 10.3 10.4 10.7 10 10.1 ...
## crimtab_long_df : 'data.frame':  3000 obs. of  2 variables:
##  $ finger: num  10 10.3 9.9 10.2 10.2 10.3 10.4 10.7 10 10.1 ...
##  $ height: num  56 57 58 58 58 58 58 58 59 59 ...
\end{verbatim}

\begin{Shaded}
\begin{Highlighting}[]
\FunctionTok{head}\NormalTok{(crimtab\_long\_df)}
\end{Highlighting}
\end{Shaded}

\begin{verbatim}
##   finger height
## 1   10.0     56
## 2   10.3     57
## 3    9.9     58
## 4   10.2     58
## 5   10.2     58
## 6   10.3     58
\end{verbatim}

\subsection{Graphic Representation}\label{graphic-representation}

\subsubsection{Base Graphics}\label{base-graphics}

\begin{itemize}
\tightlist
\item
  키와 손가락길이의 산점도
\end{itemize}

\begin{Shaded}
\begin{Highlighting}[]
\CommentTok{\# plot(finger \textasciitilde{} height, data = crimtab\_long\_df)}
\NormalTok{crimtab\_long\_df[}\DecValTok{2}\SpecialCharTok{:}\DecValTok{1}\NormalTok{] }\SpecialCharTok{\%\textgreater{}\%}
\NormalTok{  plot}
\end{Highlighting}
\end{Shaded}

\pandocbounded{\includegraphics[keepaspectratio]{crimtab_ggplots_pdf_files/figure-latex/scatter diagram-1.pdf}}

\begin{Shaded}
\begin{Highlighting}[]
\FunctionTok{plot}\NormalTok{(crimtab\_long\_df[, }\DecValTok{2}\SpecialCharTok{:}\DecValTok{1}\NormalTok{])}
\end{Highlighting}
\end{Shaded}

\begin{itemize}
\tightlist
\item
  변수 각각의 히스토그램은?
\end{itemize}

\begin{Shaded}
\begin{Highlighting}[]
\FunctionTok{par}\NormalTok{(}\AttributeTok{mfrow =} \FunctionTok{c}\NormalTok{(}\DecValTok{1}\NormalTok{, }\DecValTok{2}\NormalTok{))}
\FunctionTok{hist}\NormalTok{(crimtab\_long\_df}\SpecialCharTok{$}\NormalTok{height, }
     \AttributeTok{main =} \StringTok{"Histogram of Height"}\NormalTok{, }
     \AttributeTok{xlab =} \StringTok{"height(inches)"}\NormalTok{)}
\FunctionTok{hist}\NormalTok{(crimtab\_long\_df}\SpecialCharTok{$}\NormalTok{finger, }
     \AttributeTok{main =} \StringTok{"Histogram of Finger Length"}\NormalTok{, }
     \AttributeTok{xlab =} \StringTok{"finger length(cm)"}\NormalTok{)}
\end{Highlighting}
\end{Shaded}

\pandocbounded{\includegraphics[keepaspectratio]{crimtab_ggplots_pdf_files/figure-latex/histograms for each-1.pdf}}

\begin{Shaded}
\begin{Highlighting}[]
\CommentTok{\# hist(crimtab\_long\_df["height"], }
\CommentTok{\#      main="Histogram of Height", }
\CommentTok{\#      xlab="height(inches)")}
\CommentTok{\# hist(crimtab\_long\_df["finger"], }
\CommentTok{\#      main="Histogram of Finger Length", }
\CommentTok{\#      xlab= "finger length(cm)")}
\end{Highlighting}
\end{Shaded}

\begin{itemize}
\tightlist
\item
  평균과 표준편차를 한번에 구하려면 다음과 같이 anonymous function을
  작성하고 \texttt{mapply()} 또는 \texttt{sapply()}를 이용하는 게 편함.
  이를 모수로 하는 정규곡선을 덧씌워 볼 것.

  \begin{itemize}
  \tightlist
  \item
    \texttt{mean\_sd()}도 \texttt{anonymous\ function} 으로 평균과
    표준편차를 계산해서 출력하는 함수임. 이와 같은 함수를 저장해
    놓으려면 \texttt{dump()}를 이용함.
  \item
    이와 같이 계산한 평균과 표준편차를 모수로 하는 정규곡선을 덧씌워 볼
    것.
  \end{itemize}
\end{itemize}

\begin{Shaded}
\begin{Highlighting}[]
\NormalTok{mean\_sd }\OtherTok{\textless{}{-}} \ControlFlowTok{function}\NormalTok{(x) \{}
\NormalTok{  mean }\OtherTok{\textless{}{-}} \FunctionTok{mean}\NormalTok{(x, }\AttributeTok{na.rm =} \ConstantTok{TRUE}\NormalTok{)}
\NormalTok{  sd }\OtherTok{\textless{}{-}} \FunctionTok{sd}\NormalTok{(x)}
  \FunctionTok{c}\NormalTok{(}\AttributeTok{mean =}\NormalTok{ mean, }\AttributeTok{sd =}\NormalTok{ sd)}
\CommentTok{\# list(mean = mean, sd = sd)}
\NormalTok{\}}
\FunctionTok{dump}\NormalTok{(}\StringTok{"mean\_sd"}\NormalTok{, }\AttributeTok{file =} \StringTok{"mean\_sd.R"}\NormalTok{)}
\end{Highlighting}
\end{Shaded}

\begin{Shaded}
\begin{Highlighting}[]
\CommentTok{\# crimtab\_long\_df \%\textgreater{}\%}
\CommentTok{\#   sapply(FUN = mean\_sd)}
\NormalTok{crimtab\_stat }\OtherTok{\textless{}{-}} \FunctionTok{sapply}\NormalTok{(crimtab\_long\_df, }\AttributeTok{FUN =}\NormalTok{ mean\_sd)}
\CommentTok{\# crimtab\_stat \textless{}{-} mapply(mean\_sd, crimtab\_long\_df)}
\CommentTok{\# apply(crimtab\_long, 2, mean)}
\CommentTok{\# apply(crimtab\_long, 2, sd)}
\FunctionTok{str}\NormalTok{(crimtab\_stat)}
\end{Highlighting}
\end{Shaded}

\begin{verbatim}
##  num [1:2, 1:2] 11.547 0.549 65.473 2.558
##  - attr(*, "dimnames")=List of 2
##   ..$ : chr [1:2] "mean" "sd"
##   ..$ : chr [1:2] "finger" "height"
\end{verbatim}

\begin{itemize}
\tightlist
\item
  \texttt{crimtab\_stat}이 어떤 성격을 갖는지 다음 질문과 추출 작업을
  통해서 알아보자.
\end{itemize}

\begin{Shaded}
\begin{Highlighting}[]
\FunctionTok{is.matrix}\NormalTok{(crimtab\_stat)}
\end{Highlighting}
\end{Shaded}

\begin{verbatim}
## [1] TRUE
\end{verbatim}

\begin{Shaded}
\begin{Highlighting}[]
\FunctionTok{is.table}\NormalTok{(crimtab\_stat)}
\end{Highlighting}
\end{Shaded}

\begin{verbatim}
## [1] FALSE
\end{verbatim}

\begin{Shaded}
\begin{Highlighting}[]
\FunctionTok{is.list}\NormalTok{(crimtab\_stat)}
\end{Highlighting}
\end{Shaded}

\begin{verbatim}
## [1] FALSE
\end{verbatim}

\begin{Shaded}
\begin{Highlighting}[]
\FunctionTok{is.data.frame}\NormalTok{(crimtab\_stat)}
\end{Highlighting}
\end{Shaded}

\begin{verbatim}
## [1] FALSE
\end{verbatim}

\begin{Shaded}
\begin{Highlighting}[]
\NormalTok{crimtab\_stat[, }\DecValTok{1}\NormalTok{]}
\end{Highlighting}
\end{Shaded}

\begin{verbatim}
##       mean         sd 
## 11.5473667  0.5487137
\end{verbatim}

\begin{Shaded}
\begin{Highlighting}[]
\NormalTok{crimtab\_stat[, }\StringTok{"finger"}\NormalTok{]}
\end{Highlighting}
\end{Shaded}

\begin{verbatim}
##       mean         sd 
## 11.5473667  0.5487137
\end{verbatim}

\begin{Shaded}
\begin{Highlighting}[]
\NormalTok{crimtab\_stat[, }\StringTok{"finger"}\NormalTok{][}\DecValTok{1}\NormalTok{]}
\end{Highlighting}
\end{Shaded}

\begin{verbatim}
##     mean 
## 11.54737
\end{verbatim}

\begin{Shaded}
\begin{Highlighting}[]
\NormalTok{crimtab\_stat[, }\StringTok{"finger"}\NormalTok{][[}\DecValTok{1}\NormalTok{]]}
\end{Highlighting}
\end{Shaded}

\begin{verbatim}
## [1] 11.54737
\end{verbatim}

\begin{Shaded}
\begin{Highlighting}[]
\NormalTok{crimtab\_stat[}\DecValTok{1}\NormalTok{]}
\end{Highlighting}
\end{Shaded}

\begin{verbatim}
## [1] 11.54737
\end{verbatim}

\begin{Shaded}
\begin{Highlighting}[]
\NormalTok{crimtab\_stat[}\DecValTok{2}\SpecialCharTok{:}\DecValTok{3}\NormalTok{]}
\end{Highlighting}
\end{Shaded}

\begin{verbatim}
## [1]  0.5487137 65.4730000
\end{verbatim}

\begin{Shaded}
\begin{Highlighting}[]
\CommentTok{\# crimtab\_stat["finger"]}
\CommentTok{\# crimtab\_stat$finger}
\end{Highlighting}
\end{Shaded}

\texttt{matrix} 를 \texttt{data\ frame} 으로 변환하면

\begin{Shaded}
\begin{Highlighting}[]
\NormalTok{(crimtab\_stat\_df }\OtherTok{\textless{}{-}} \FunctionTok{data.frame}\NormalTok{(crimtab\_stat))}
\end{Highlighting}
\end{Shaded}

\begin{verbatim}
##          finger    height
## mean 11.5473667 65.473000
## sd    0.5487137  2.557757
\end{verbatim}

\begin{Shaded}
\begin{Highlighting}[]
\FunctionTok{is.matrix}\NormalTok{(crimtab\_stat\_df)}
\end{Highlighting}
\end{Shaded}

\begin{verbatim}
## [1] FALSE
\end{verbatim}

\begin{Shaded}
\begin{Highlighting}[]
\FunctionTok{is.table}\NormalTok{(crimtab\_stat\_df)}
\end{Highlighting}
\end{Shaded}

\begin{verbatim}
## [1] FALSE
\end{verbatim}

\begin{Shaded}
\begin{Highlighting}[]
\FunctionTok{is.list}\NormalTok{(crimtab\_stat\_df)}
\end{Highlighting}
\end{Shaded}

\begin{verbatim}
## [1] TRUE
\end{verbatim}

\begin{Shaded}
\begin{Highlighting}[]
\FunctionTok{is.data.frame}\NormalTok{(crimtab\_stat\_df)}
\end{Highlighting}
\end{Shaded}

\begin{verbatim}
## [1] TRUE
\end{verbatim}

\begin{Shaded}
\begin{Highlighting}[]
\NormalTok{crimtab\_stat\_df[, }\DecValTok{1}\NormalTok{]}
\end{Highlighting}
\end{Shaded}

\begin{verbatim}
## [1] 11.5473667  0.5487137
\end{verbatim}

\begin{Shaded}
\begin{Highlighting}[]
\FunctionTok{str}\NormalTok{(crimtab\_stat\_df[, }\DecValTok{1}\NormalTok{])}
\end{Highlighting}
\end{Shaded}

\begin{verbatim}
##  num [1:2] 11.547 0.549
\end{verbatim}

\begin{Shaded}
\begin{Highlighting}[]
\NormalTok{crimtab\_stat\_df[, }\StringTok{"finger"}\NormalTok{]}
\end{Highlighting}
\end{Shaded}

\begin{verbatim}
## [1] 11.5473667  0.5487137
\end{verbatim}

\begin{Shaded}
\begin{Highlighting}[]
\FunctionTok{str}\NormalTok{(crimtab\_stat\_df[, }\StringTok{"finger"}\NormalTok{])}
\end{Highlighting}
\end{Shaded}

\begin{verbatim}
##  num [1:2] 11.547 0.549
\end{verbatim}

\begin{Shaded}
\begin{Highlighting}[]
\NormalTok{crimtab\_stat\_df[, }\StringTok{"finger"}\NormalTok{][}\DecValTok{1}\NormalTok{]}
\end{Highlighting}
\end{Shaded}

\begin{verbatim}
## [1] 11.54737
\end{verbatim}

\begin{Shaded}
\begin{Highlighting}[]
\FunctionTok{str}\NormalTok{(crimtab\_stat\_df[, }\StringTok{"finger"}\NormalTok{][}\DecValTok{1}\NormalTok{])}
\end{Highlighting}
\end{Shaded}

\begin{verbatim}
##  num 11.5
\end{verbatim}

\begin{Shaded}
\begin{Highlighting}[]
\NormalTok{crimtab\_stat\_df[, }\StringTok{"finger"}\NormalTok{][[}\DecValTok{1}\NormalTok{]]}
\end{Highlighting}
\end{Shaded}

\begin{verbatim}
## [1] 11.54737
\end{verbatim}

\begin{Shaded}
\begin{Highlighting}[]
\FunctionTok{str}\NormalTok{(crimtab\_stat\_df[, }\StringTok{"finger"}\NormalTok{][[}\DecValTok{1}\NormalTok{]])}
\end{Highlighting}
\end{Shaded}

\begin{verbatim}
##  num 11.5
\end{verbatim}

\begin{Shaded}
\begin{Highlighting}[]
\NormalTok{crimtab\_stat\_df[}\DecValTok{1}\NormalTok{]}
\end{Highlighting}
\end{Shaded}

\begin{verbatim}
##          finger
## mean 11.5473667
## sd    0.5487137
\end{verbatim}

\begin{Shaded}
\begin{Highlighting}[]
\FunctionTok{str}\NormalTok{(crimtab\_stat\_df[}\DecValTok{1}\NormalTok{])}
\end{Highlighting}
\end{Shaded}

\begin{verbatim}
## 'data.frame':    2 obs. of  1 variable:
##  $ finger: num  11.547 0.549
\end{verbatim}

\begin{Shaded}
\begin{Highlighting}[]
\NormalTok{crimtab\_stat\_df[}\StringTok{"finger"}\NormalTok{]}
\end{Highlighting}
\end{Shaded}

\begin{verbatim}
##          finger
## mean 11.5473667
## sd    0.5487137
\end{verbatim}

\begin{Shaded}
\begin{Highlighting}[]
\FunctionTok{str}\NormalTok{(crimtab\_stat\_df[}\StringTok{"finger"}\NormalTok{])}
\end{Highlighting}
\end{Shaded}

\begin{verbatim}
## 'data.frame':    2 obs. of  1 variable:
##  $ finger: num  11.547 0.549
\end{verbatim}

\begin{Shaded}
\begin{Highlighting}[]
\NormalTok{crimtab\_stat\_df[}\StringTok{"finger"}\NormalTok{][}\DecValTok{1}\NormalTok{]}
\end{Highlighting}
\end{Shaded}

\begin{verbatim}
##          finger
## mean 11.5473667
## sd    0.5487137
\end{verbatim}

\begin{Shaded}
\begin{Highlighting}[]
\FunctionTok{str}\NormalTok{(crimtab\_stat\_df[}\StringTok{"finger"}\NormalTok{][}\DecValTok{1}\NormalTok{])}
\end{Highlighting}
\end{Shaded}

\begin{verbatim}
## 'data.frame':    2 obs. of  1 variable:
##  $ finger: num  11.547 0.549
\end{verbatim}

\begin{Shaded}
\begin{Highlighting}[]
\NormalTok{crimtab\_stat\_df[}\StringTok{"finger"}\NormalTok{][[}\DecValTok{1}\NormalTok{]]}
\end{Highlighting}
\end{Shaded}

\begin{verbatim}
## [1] 11.5473667  0.5487137
\end{verbatim}

\begin{Shaded}
\begin{Highlighting}[]
\FunctionTok{str}\NormalTok{(crimtab\_stat\_df[}\StringTok{"finger"}\NormalTok{][[}\DecValTok{1}\NormalTok{]])}
\end{Highlighting}
\end{Shaded}

\begin{verbatim}
##  num [1:2] 11.547 0.549
\end{verbatim}

\begin{Shaded}
\begin{Highlighting}[]
\NormalTok{crimtab\_stat\_df}\SpecialCharTok{$}\NormalTok{finger}
\end{Highlighting}
\end{Shaded}

\begin{verbatim}
## [1] 11.5473667  0.5487137
\end{verbatim}

\begin{Shaded}
\begin{Highlighting}[]
\FunctionTok{str}\NormalTok{(crimtab\_stat\_df}\SpecialCharTok{$}\NormalTok{finger)}
\end{Highlighting}
\end{Shaded}

\begin{verbatim}
##  num [1:2] 11.547 0.549
\end{verbatim}

\begin{Shaded}
\begin{Highlighting}[]
\NormalTok{crimtab\_stat\_df}\SpecialCharTok{$}\NormalTok{finger[}\DecValTok{1}\NormalTok{]}
\end{Highlighting}
\end{Shaded}

\begin{verbatim}
## [1] 11.54737
\end{verbatim}

\begin{Shaded}
\begin{Highlighting}[]
\FunctionTok{str}\NormalTok{(crimtab\_stat\_df}\SpecialCharTok{$}\NormalTok{finger[}\DecValTok{1}\NormalTok{])}
\end{Highlighting}
\end{Shaded}

\begin{verbatim}
##  num 11.5
\end{verbatim}

\begin{Shaded}
\begin{Highlighting}[]
\NormalTok{crimtab\_stat\_df}\SpecialCharTok{$}\NormalTok{finger[[}\DecValTok{1}\NormalTok{]]}
\end{Highlighting}
\end{Shaded}

\begin{verbatim}
## [1] 11.54737
\end{verbatim}

\begin{Shaded}
\begin{Highlighting}[]
\FunctionTok{str}\NormalTok{(crimtab\_stat\_df}\SpecialCharTok{$}\NormalTok{finger[[}\DecValTok{1}\NormalTok{]])}
\end{Highlighting}
\end{Shaded}

\begin{verbatim}
##  num 11.5
\end{verbatim}

\subsubsection{ggplot}\label{ggplot}

\begin{itemize}
\tightlist
\item
  키와 손가락 길이의 산점도
\end{itemize}

\begin{Shaded}
\begin{Highlighting}[]
\FunctionTok{library}\NormalTok{(ggplot2)}
\NormalTok{g1 }\OtherTok{\textless{}{-}} \FunctionTok{ggplot}\NormalTok{(}\AttributeTok{data =}\NormalTok{ crimtab\_long\_df, }
             \AttributeTok{mapping =} \FunctionTok{aes}\NormalTok{(}\AttributeTok{x =}\NormalTok{ height, }\AttributeTok{y =}\NormalTok{ finger)) }
\NormalTok{g2 }\OtherTok{\textless{}{-}}\NormalTok{ g1 }\SpecialCharTok{+} 
  \FunctionTok{geom\_point}\NormalTok{()}
\NormalTok{g2}
\end{Highlighting}
\end{Shaded}

\pandocbounded{\includegraphics[keepaspectratio]{crimtab_ggplots_pdf_files/figure-latex/scatter of finger and height-1.pdf}}

\begin{itemize}
\tightlist
\item
  투명도 변경 : \texttt{alpha\ =\ 0.9}
\end{itemize}

\begin{Shaded}
\begin{Highlighting}[]
\NormalTok{g2\_2 }\OtherTok{\textless{}{-}}\NormalTok{ g1 }\SpecialCharTok{+} 
  \FunctionTok{geom\_point}\NormalTok{(}\AttributeTok{alpha =} \FloatTok{0.9}\NormalTok{)}
\NormalTok{g2\_2}
\end{Highlighting}
\end{Shaded}

\pandocbounded{\includegraphics[keepaspectratio]{crimtab_ggplots_pdf_files/figure-latex/alpha variation 0.9-1.pdf}}

\begin{itemize}
\tightlist
\item
  투명도 변경 : \texttt{alpha\ =\ 0.5}
\end{itemize}

\begin{Shaded}
\begin{Highlighting}[]
\NormalTok{g2\_3 }\OtherTok{\textless{}{-}}\NormalTok{ g1 }\SpecialCharTok{+} 
  \FunctionTok{geom\_point}\NormalTok{(}\AttributeTok{alpha =} \FloatTok{0.5}\NormalTok{)}
\NormalTok{g2\_3}
\end{Highlighting}
\end{Shaded}

\pandocbounded{\includegraphics[keepaspectratio]{crimtab_ggplots_pdf_files/figure-latex/alpha variation 0.5-1.pdf}}

\begin{itemize}
\tightlist
\item
  투명도 변경 : \texttt{alpha\ =\ 0.1}
\end{itemize}

\begin{Shaded}
\begin{Highlighting}[]
\NormalTok{g2\_4 }\OtherTok{\textless{}{-}}\NormalTok{ g1 }\SpecialCharTok{+} 
  \FunctionTok{geom\_point}\NormalTok{(}\AttributeTok{alpha =} \FloatTok{0.1}\NormalTok{)}
\NormalTok{g2\_4}
\end{Highlighting}
\end{Shaded}

\pandocbounded{\includegraphics[keepaspectratio]{crimtab_ggplots_pdf_files/figure-latex/alpha variation 0.1-1.pdf}}

\begin{itemize}
\tightlist
\item
  중복점 흐트러놓기 : \texttt{position\ =\ jitter}
\end{itemize}

\begin{Shaded}
\begin{Highlighting}[]
\NormalTok{g2\_5 }\OtherTok{\textless{}{-}}\NormalTok{ g1 }\SpecialCharTok{+} 
  \FunctionTok{geom\_point}\NormalTok{(}\AttributeTok{position =} \StringTok{"jitter"}\NormalTok{)}
\NormalTok{g2\_5}
\end{Highlighting}
\end{Shaded}

\pandocbounded{\includegraphics[keepaspectratio]{crimtab_ggplots_pdf_files/figure-latex/position jitter-1.pdf}}

\begin{itemize}
\tightlist
\item
  점의 크기를 줄이고 중복점 흐트러놓기 :
  \texttt{position\ =\ jitter,\ size\ =\ 0.7}
\end{itemize}

\begin{Shaded}
\begin{Highlighting}[]
\NormalTok{g2\_6 }\OtherTok{\textless{}{-}}\NormalTok{ g1 }\SpecialCharTok{+} 
  \FunctionTok{geom\_point}\NormalTok{(}\AttributeTok{position =} \StringTok{"jitter"}\NormalTok{, }\AttributeTok{size =} \FloatTok{0.7}\NormalTok{)}
\NormalTok{g2\_6}
\end{Highlighting}
\end{Shaded}

\pandocbounded{\includegraphics[keepaspectratio]{crimtab_ggplots_pdf_files/figure-latex/position jitter size-1.pdf}}

\begin{itemize}
\tightlist
\item
  동일한 효과 : \texttt{position\ =\ position\_jitter(),\ size\ =\ 0.7}
\end{itemize}

\begin{Shaded}
\begin{Highlighting}[]
\NormalTok{g2\_7 }\OtherTok{\textless{}{-}}\NormalTok{ g1 }\SpecialCharTok{+} 
  \FunctionTok{geom\_point}\NormalTok{(}\AttributeTok{position =} \FunctionTok{position\_jitter}\NormalTok{(), }\AttributeTok{size =} \FloatTok{0.7}\NormalTok{)}
\NormalTok{g2\_7}
\end{Highlighting}
\end{Shaded}

\pandocbounded{\includegraphics[keepaspectratio]{crimtab_ggplots_pdf_files/figure-latex/position jitter size alt-1.pdf}}

\begin{itemize}
\tightlist
\item
  흐트러놓는 폭 조절 :
  \texttt{width\ =\ 1,\ height\ =\ 0,\ size\ =\ 0.7}
\end{itemize}

\begin{Shaded}
\begin{Highlighting}[]
\NormalTok{g2\_8 }\OtherTok{\textless{}{-}}\NormalTok{ g1 }\SpecialCharTok{+} 
  \FunctionTok{geom\_point}\NormalTok{(}\AttributeTok{position =} \FunctionTok{position\_jitter}\NormalTok{(}\AttributeTok{width =} \DecValTok{1}\NormalTok{, }\AttributeTok{height =} \DecValTok{0}\NormalTok{), }
             \AttributeTok{size =} \FloatTok{0.7}\NormalTok{)}
\NormalTok{g2\_8}
\end{Highlighting}
\end{Shaded}

\pandocbounded{\includegraphics[keepaspectratio]{crimtab_ggplots_pdf_files/figure-latex/position jitter size width-1.pdf}}

\begin{itemize}
\tightlist
\item
  흐트러놓는 폭과 높이 조절 :
  \texttt{width\ =\ 1,\ height\ =\ 0.1,\ size\ =\ 0.7}
\end{itemize}

\begin{Shaded}
\begin{Highlighting}[]
\NormalTok{g2\_9 }\OtherTok{\textless{}{-}}\NormalTok{ g1 }\SpecialCharTok{+} 
  \FunctionTok{geom\_point}\NormalTok{(}\AttributeTok{position =} \FunctionTok{position\_jitter}\NormalTok{(}\AttributeTok{width =} \DecValTok{1}\NormalTok{, }\AttributeTok{height =} \FloatTok{0.1}\NormalTok{), }
             \AttributeTok{size =} \FloatTok{0.7}\NormalTok{)}
\NormalTok{g2\_9}
\end{Highlighting}
\end{Shaded}

\pandocbounded{\includegraphics[keepaspectratio]{crimtab_ggplots_pdf_files/figure-latex/position jitter size width height-1.pdf}}

\begin{itemize}
\tightlist
\item
  흑백 테마 : \texttt{theme\_bw()}
\end{itemize}

\begin{Shaded}
\begin{Highlighting}[]
\NormalTok{g3 }\OtherTok{\textless{}{-}}\NormalTok{ g2\_9 }\SpecialCharTok{+}
  \FunctionTok{theme\_bw}\NormalTok{()}
\NormalTok{g3}
\end{Highlighting}
\end{Shaded}

\pandocbounded{\includegraphics[keepaspectratio]{crimtab_ggplots_pdf_files/figure-latex/position jitter size bw-1.pdf}}

\subsubsection{히스토그램}\label{uxd788uxc2a4uxd1a0uxadf8uxb7a8}

\begin{Shaded}
\begin{Highlighting}[]
\NormalTok{h1 }\OtherTok{\textless{}{-}} \FunctionTok{ggplot}\NormalTok{(}\AttributeTok{data =}\NormalTok{ crimtab\_long\_df, }
             \AttributeTok{mapping =} \FunctionTok{aes}\NormalTok{(}\AttributeTok{x =}\NormalTok{ height)) }
\NormalTok{h1 }\SpecialCharTok{+} \FunctionTok{geom\_histogram}\NormalTok{(}\AttributeTok{alpha =} \FloatTok{0.5}\NormalTok{)}
\end{Highlighting}
\end{Shaded}

\begin{verbatim}
## `stat_bin()` using `bins = 30`. Pick better value with `binwidth`.
\end{verbatim}

\pandocbounded{\includegraphics[keepaspectratio]{crimtab_ggplots_pdf_files/figure-latex/unnamed-chunk-1-1.pdf}}

\begin{Shaded}
\begin{Highlighting}[]
\NormalTok{f1 }\OtherTok{\textless{}{-}} \FunctionTok{ggplot}\NormalTok{(}\AttributeTok{data =}\NormalTok{ crimtab\_long\_df, }
             \AttributeTok{mapping =} \FunctionTok{aes}\NormalTok{(}\AttributeTok{x =}\NormalTok{ finger))}
\NormalTok{f1 }\SpecialCharTok{+} \FunctionTok{geom\_histogram}\NormalTok{(}\AttributeTok{alpha =} \FloatTok{0.5}\NormalTok{)}
\end{Highlighting}
\end{Shaded}

\begin{verbatim}
## `stat_bin()` using `bins = 30`. Pick better value with `binwidth`.
\end{verbatim}

\pandocbounded{\includegraphics[keepaspectratio]{crimtab_ggplots_pdf_files/figure-latex/unnamed-chunk-2-1.pdf}}

\begin{Shaded}
\begin{Highlighting}[]
\NormalTok{h1 }\SpecialCharTok{+} \FunctionTok{geom\_histogram}\NormalTok{(}\FunctionTok{aes}\NormalTok{(}\AttributeTok{y =} \FunctionTok{after\_stat}\NormalTok{(density)),}
                    \AttributeTok{binwidth =} \DecValTok{1}\NormalTok{, }
                    \AttributeTok{alpha =} \FloatTok{0.5}\NormalTok{)}
\end{Highlighting}
\end{Shaded}

\pandocbounded{\includegraphics[keepaspectratio]{crimtab_ggplots_pdf_files/figure-latex/unnamed-chunk-3-1.pdf}}

\begin{Shaded}
\begin{Highlighting}[]
\NormalTok{f1 }\SpecialCharTok{+} \FunctionTok{geom\_histogram}\NormalTok{(}\FunctionTok{aes}\NormalTok{(}\AttributeTok{y =} \FunctionTok{after\_stat}\NormalTok{(density)), }
                    \AttributeTok{binwidth =} \FloatTok{0.1}\NormalTok{, }
                    \AttributeTok{alpha =} \FloatTok{0.5}\NormalTok{)}
\end{Highlighting}
\end{Shaded}

\pandocbounded{\includegraphics[keepaspectratio]{crimtab_ggplots_pdf_files/figure-latex/unnamed-chunk-4-1.pdf}}

\begin{Shaded}
\begin{Highlighting}[]
\NormalTok{(g\_h\_1 }\OtherTok{\textless{}{-}}\NormalTok{ h1 }\SpecialCharTok{+} 
   \FunctionTok{geom\_histogram}\NormalTok{(}\FunctionTok{aes}\NormalTok{(}\AttributeTok{y =} \FunctionTok{after\_stat}\NormalTok{(density)), }
                  \AttributeTok{binwidth =} \DecValTok{1}\NormalTok{, }
                  \AttributeTok{fill =} \StringTok{"white"}\NormalTok{, }
                  \AttributeTok{colour =} \StringTok{"black"}\NormalTok{))}
\end{Highlighting}
\end{Shaded}

\pandocbounded{\includegraphics[keepaspectratio]{crimtab_ggplots_pdf_files/figure-latex/unnamed-chunk-5-1.pdf}}

\begin{Shaded}
\begin{Highlighting}[]
\NormalTok{(g\_h }\OtherTok{\textless{}{-}}\NormalTok{ g\_h\_1 }\SpecialCharTok{+}
   \FunctionTok{theme\_bw}\NormalTok{() }\SpecialCharTok{+}
   \FunctionTok{scale\_x\_continuous}\NormalTok{(}\AttributeTok{name =} \StringTok{""}\NormalTok{, }
                      \AttributeTok{breaks =} \ConstantTok{NULL}\NormalTok{) }\SpecialCharTok{+}
   \FunctionTok{scale\_y\_continuous}\NormalTok{(}\AttributeTok{name =} \StringTok{""}\NormalTok{, }
                      \AttributeTok{breaks =} \ConstantTok{NULL}\NormalTok{))}
\end{Highlighting}
\end{Shaded}

\pandocbounded{\includegraphics[keepaspectratio]{crimtab_ggplots_pdf_files/figure-latex/unnamed-chunk-6-1.pdf}}

\begin{Shaded}
\begin{Highlighting}[]
\NormalTok{(g\_f\_1 }\OtherTok{\textless{}{-}}\NormalTok{ f1 }\SpecialCharTok{+} 
   \FunctionTok{geom\_histogram}\NormalTok{(}\FunctionTok{aes}\NormalTok{(}\AttributeTok{y =} \FunctionTok{after\_stat}\NormalTok{(density)), }
                  \AttributeTok{binwidth =} \FloatTok{0.2}\NormalTok{, }
                  \AttributeTok{fill =} \StringTok{"white"}\NormalTok{, }
                  \AttributeTok{colour =} \StringTok{"black"}\NormalTok{))}
\end{Highlighting}
\end{Shaded}

\pandocbounded{\includegraphics[keepaspectratio]{crimtab_ggplots_pdf_files/figure-latex/unnamed-chunk-7-1.pdf}}

\begin{Shaded}
\begin{Highlighting}[]
\NormalTok{(g\_f }\OtherTok{\textless{}{-}}\NormalTok{ g\_f\_1 }\SpecialCharTok{+}
   \FunctionTok{theme\_bw}\NormalTok{() }\SpecialCharTok{+}
   \FunctionTok{scale\_x\_continuous}\NormalTok{(}\AttributeTok{name =} \StringTok{""}\NormalTok{, }
                      \AttributeTok{breaks =} \ConstantTok{NULL}\NormalTok{) }\SpecialCharTok{+}
   \FunctionTok{scale\_y\_continuous}\NormalTok{(}\AttributeTok{name =} \StringTok{""}\NormalTok{, }
                      \AttributeTok{breaks =} \ConstantTok{NULL}\NormalTok{) }\SpecialCharTok{+}
   \FunctionTok{coord\_flip}\NormalTok{())}
\end{Highlighting}
\end{Shaded}

\pandocbounded{\includegraphics[keepaspectratio]{crimtab_ggplots_pdf_files/figure-latex/unnamed-chunk-8-1.pdf}}

\subsubsection{평균 위치를 화살표로
나타내려면}\label{uxd3c9uxade0-uxc704uxce58uxb97c-uxd654uxc0b4uxd45cuxb85c-uxb098uxd0c0uxb0b4uxb824uxba74}

\begin{Shaded}
\begin{Highlighting}[]
\FunctionTok{library}\NormalTok{(grid)}
\NormalTok{(mean\_finger }\OtherTok{\textless{}{-}}\NormalTok{ crimtab\_stat[, }\DecValTok{1}\NormalTok{][[}\DecValTok{1}\NormalTok{]])}
\end{Highlighting}
\end{Shaded}

\begin{verbatim}
## [1] 11.54737
\end{verbatim}

\begin{Shaded}
\begin{Highlighting}[]
\NormalTok{(sd\_finger }\OtherTok{\textless{}{-}}\NormalTok{ crimtab\_stat[, }\DecValTok{1}\NormalTok{][[}\DecValTok{2}\NormalTok{]])}
\end{Highlighting}
\end{Shaded}

\begin{verbatim}
## [1] 0.5487137
\end{verbatim}

\begin{Shaded}
\begin{Highlighting}[]
\NormalTok{(mean\_height }\OtherTok{\textless{}{-}}\NormalTok{ crimtab\_stat[, }\DecValTok{2}\NormalTok{][[}\DecValTok{1}\NormalTok{]])}
\end{Highlighting}
\end{Shaded}

\begin{verbatim}
## [1] 65.473
\end{verbatim}

\begin{Shaded}
\begin{Highlighting}[]
\NormalTok{(sd\_height }\OtherTok{\textless{}{-}}\NormalTok{ crimtab\_stat[, }\DecValTok{2}\NormalTok{][[}\DecValTok{2}\NormalTok{]])}
\end{Highlighting}
\end{Shaded}

\begin{verbatim}
## [1] 2.557757
\end{verbatim}

\begin{Shaded}
\begin{Highlighting}[]
\NormalTok{x\_finger }\OtherTok{\textless{}{-}} \FunctionTok{seq}\NormalTok{(}\FloatTok{9.5}\NormalTok{, }\FloatTok{13.5}\NormalTok{, }
                \AttributeTok{length.out =} \DecValTok{3000}\NormalTok{)}
\NormalTok{y\_finger }\OtherTok{\textless{}{-}} \FunctionTok{dnorm}\NormalTok{(x\_finger, }
                  \AttributeTok{mean =}\NormalTok{ mean\_finger, }
                  \AttributeTok{sd =}\NormalTok{ sd\_finger)}
\NormalTok{x\_height }\OtherTok{\textless{}{-}} \FunctionTok{seq}\NormalTok{(}\DecValTok{56}\NormalTok{, }\DecValTok{77}\NormalTok{, }
                \AttributeTok{length.out =} \DecValTok{3000}\NormalTok{)}
\NormalTok{y\_height }\OtherTok{\textless{}{-}} \FunctionTok{dnorm}\NormalTok{(x\_height, }
                  \AttributeTok{mean =}\NormalTok{ mean\_height, }
                  \AttributeTok{sd =}\NormalTok{ sd\_height)}
\end{Highlighting}
\end{Shaded}

\begin{Shaded}
\begin{Highlighting}[]
\NormalTok{(g\_h\_2 }\OtherTok{\textless{}{-}}\NormalTok{ g\_h\_1 }\SpecialCharTok{+} 
    \FunctionTok{annotate}\NormalTok{(}\StringTok{"segment"}\NormalTok{, }
             \AttributeTok{x =}\NormalTok{ mean\_height, }
             \AttributeTok{xend =}\NormalTok{ mean\_height, }
             \AttributeTok{y =} \SpecialCharTok{{-}}\FloatTok{0.02}\NormalTok{, }
             \AttributeTok{yend =} \DecValTok{0}\NormalTok{, }
             \AttributeTok{arrow =} \FunctionTok{arrow}\NormalTok{(}\AttributeTok{length =} \FunctionTok{unit}\NormalTok{(}\FloatTok{0.3}\NormalTok{, }\StringTok{"cm"}\NormalTok{))))}
\end{Highlighting}
\end{Shaded}

\pandocbounded{\includegraphics[keepaspectratio]{crimtab_ggplots_pdf_files/figure-latex/unnamed-chunk-9-1.pdf}}

\begin{Shaded}
\begin{Highlighting}[]
\NormalTok{(g\_h\_3 }\OtherTok{\textless{}{-}}\NormalTok{ g\_h\_2 }\SpecialCharTok{+} 
   \FunctionTok{theme\_bw}\NormalTok{())}
\end{Highlighting}
\end{Shaded}

\pandocbounded{\includegraphics[keepaspectratio]{crimtab_ggplots_pdf_files/figure-latex/unnamed-chunk-10-1.pdf}}

\begin{Shaded}
\begin{Highlighting}[]
\NormalTok{(g\_h\_4 }\OtherTok{\textless{}{-}}\NormalTok{ g\_h\_3 }\SpecialCharTok{+} 
   \FunctionTok{geom\_line}\NormalTok{(}\FunctionTok{aes}\NormalTok{(}\AttributeTok{x =}\NormalTok{ x\_height, }\AttributeTok{y =}\NormalTok{ y\_height)))}
\end{Highlighting}
\end{Shaded}

\pandocbounded{\includegraphics[keepaspectratio]{crimtab_ggplots_pdf_files/figure-latex/unnamed-chunk-11-1.pdf}}

\begin{Shaded}
\begin{Highlighting}[]
\NormalTok{(g\_f\_2 }\OtherTok{\textless{}{-}}\NormalTok{ g\_f\_1 }\SpecialCharTok{+} 
   \FunctionTok{annotate}\NormalTok{(}\StringTok{"segment"}\NormalTok{, }
            \AttributeTok{x =}\NormalTok{ mean\_finger, }
            \AttributeTok{xend =}\NormalTok{ mean\_finger, }
            \AttributeTok{y =} \SpecialCharTok{{-}}\FloatTok{0.1}\NormalTok{, }
            \AttributeTok{yend =} \DecValTok{0}\NormalTok{, }
            \AttributeTok{arrow =} \FunctionTok{arrow}\NormalTok{(}\AttributeTok{length =} \FunctionTok{unit}\NormalTok{(}\FloatTok{0.3}\NormalTok{, }\StringTok{"cm"}\NormalTok{))))}
\end{Highlighting}
\end{Shaded}

\pandocbounded{\includegraphics[keepaspectratio]{crimtab_ggplots_pdf_files/figure-latex/unnamed-chunk-12-1.pdf}}

\begin{Shaded}
\begin{Highlighting}[]
\NormalTok{(g\_f\_3 }\OtherTok{\textless{}{-}}\NormalTok{ g\_f\_2 }\SpecialCharTok{+} 
   \FunctionTok{theme\_bw}\NormalTok{())}
\end{Highlighting}
\end{Shaded}

\pandocbounded{\includegraphics[keepaspectratio]{crimtab_ggplots_pdf_files/figure-latex/unnamed-chunk-13-1.pdf}}

\begin{Shaded}
\begin{Highlighting}[]
\NormalTok{(g\_f\_4 }\OtherTok{\textless{}{-}}\NormalTok{ g\_f\_3 }\SpecialCharTok{+} 
   \FunctionTok{geom\_line}\NormalTok{(}\FunctionTok{aes}\NormalTok{(}\AttributeTok{x =}\NormalTok{ x\_finger, }\AttributeTok{y =}\NormalTok{ y\_finger)))}
\end{Highlighting}
\end{Shaded}

\pandocbounded{\includegraphics[keepaspectratio]{crimtab_ggplots_pdf_files/figure-latex/unnamed-chunk-14-1.pdf}}

\subsubsection{산점도와 히스토그램 함께
배열하기}\label{uxc0b0uxc810uxb3c4uxc640-uxd788uxc2a4uxd1a0uxadf8uxb7a8-uxd568uxaed8-uxbc30uxc5f4uxd558uxae30}

\texttt{grid} 및 \texttt{gridExtra} 패키지와 함께 \texttt{blank} Grob
설정이 핵심. \texttt{grid.arrange} 사용법에 유의.

\begin{Shaded}
\begin{Highlighting}[]
\FunctionTok{library}\NormalTok{(gridExtra)}
\FunctionTok{grid.rect}\NormalTok{(}\AttributeTok{gp =} \FunctionTok{gpar}\NormalTok{(}\AttributeTok{col =} \StringTok{"white"}\NormalTok{), }\AttributeTok{draw =} \ConstantTok{FALSE}\NormalTok{) }\SpecialCharTok{\%\textgreater{}\%}
\FunctionTok{grid.arrange}\NormalTok{(g\_h, ., g3, g\_f, }
             \AttributeTok{ncol =} \DecValTok{2}\NormalTok{, }
             \AttributeTok{widths =} \FunctionTok{c}\NormalTok{(}\DecValTok{4}\NormalTok{, }\DecValTok{1}\NormalTok{), }
             \AttributeTok{heights =} \FunctionTok{c}\NormalTok{(}\DecValTok{1}\NormalTok{, }\DecValTok{4}\NormalTok{))}
\end{Highlighting}
\end{Shaded}

\pandocbounded{\includegraphics[keepaspectratio]{crimtab_ggplots_pdf_files/figure-latex/side by side-1.pdf}}

\begin{Shaded}
\begin{Highlighting}[]
\NormalTok{blank }\OtherTok{\textless{}{-}} \FunctionTok{grid.rect}\NormalTok{(}\AttributeTok{gp =} \FunctionTok{gpar}\NormalTok{(}\AttributeTok{col =} \StringTok{"white"}\NormalTok{), }\AttributeTok{draw =} \ConstantTok{FALSE}\NormalTok{)}
\FunctionTok{grid.arrange}\NormalTok{(g\_h, blank, g3, g\_f, }
             \AttributeTok{ncol =} \DecValTok{2}\NormalTok{, }
             \AttributeTok{widths =} \FunctionTok{c}\NormalTok{(}\DecValTok{4}\NormalTok{, }\DecValTok{1}\NormalTok{), }
             \AttributeTok{heights =} \FunctionTok{c}\NormalTok{(}\DecValTok{1}\NormalTok{, }\DecValTok{4}\NormalTok{))}
\end{Highlighting}
\end{Shaded}

\pandocbounded{\includegraphics[keepaspectratio]{crimtab_ggplots_pdf_files/figure-latex/side by side-2.pdf}}

\begin{Shaded}
\begin{Highlighting}[]
\NormalTok{blank }\OtherTok{\textless{}{-}} \FunctionTok{nullGrob}\NormalTok{()}
\FunctionTok{grid.arrange}\NormalTok{(g\_h, blank, g3, g\_f, }
             \AttributeTok{ncol =} \DecValTok{2}\NormalTok{, }
             \AttributeTok{widths =} \FunctionTok{c}\NormalTok{(}\DecValTok{4}\NormalTok{, }\DecValTok{1}\NormalTok{), }
             \AttributeTok{heights =} \FunctionTok{c}\NormalTok{(}\DecValTok{1}\NormalTok{, }\DecValTok{4}\NormalTok{))}
\end{Highlighting}
\end{Shaded}

\pandocbounded{\includegraphics[keepaspectratio]{crimtab_ggplots_pdf_files/figure-latex/side by side-3.pdf}}

\end{document}
